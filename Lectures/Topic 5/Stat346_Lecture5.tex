\documentclass[compress,mathserif]{beamer}

\usepackage{amssymb}
\usepackage{amsfonts}
\usepackage{amsmath}
\usepackage{amsxtra}
\usepackage{hyperref}
\usepackage{tensor}
\usepackage{dsfont}

\mode<presentation> {
  \usetheme{default}
  \useoutertheme{infolines}
  \setbeamercovered{transparent}
  \beamertemplateballitem
}

\title[Coverage Modifications]{Frequency, Severity, and Aggregate Losses with Coverage Modifications}
\author{Chapter 8}
\institute[Stat 346]{Stat 346 - Short-term Actuarial Math}
\date[BYU]{}
\subject{Stat 346}

\begin{document}

\begin{frame}
 \titlepage
\end{frame}

\section{Introduction}
\frame{\frametitle{Introduction}
\begin{itemize}
\item In the previous weeks, we have assumed that the loss amount $X$ is also the claim amount paid.
\smallskip
\item However, there are policy modifications for which the insurer may only be liable for a portion of this loss amount.
\smallskip
\item For example:
\begin{itemize}
\item deductibles
\smallskip
\item policy limits
\smallskip
\item coinsurance
\end{itemize}
\smallskip
\item For purposes of notation, we shall denote the modified loss amount to be $Y$ and will be referred to as the \alert{claim amount} paid by the insurer. More precisely, $Y^L$ denotes the \alert{per-loss} variable while $Y^P$ denotes the \alert{per-payment} variable.
\smallskip
\item We shall refer to $X$ as the loss amount random variable.
\end{itemize}
}
%
%\section{Ordinary deductible}
%\subsection{per-loss variable}
%\frame{\frametitle{Policy deductible: per-loss variable}
%\begin{itemize}
%\item A policy (ordinary) deductible, $d$, is a threshold amount for which it must first be exceeded by a loss before any claim can be paid.
%\smallskip
%\item Once the loss $X$ exceeds this threshold, the claims paid by the insurer is $X-d$.
%\smallskip
%\item The \alert{per-loss} random variable is therefore
%\begin{equation*}
%Y^L = (X-d)_+ = \left\{
%\begin{array}{ll}
%0, & \text{for } X \leq d \\
%X-d, & \text{for } X>d
%\end{array}
%\right. .
%\end{equation*}
%where $(X-d)_+$ denotes ``the excess of $X$ over $d$, if positive''.
%\smallskip
%\item $Y^L$ is a mixed random variable where it has a probability mass at 0 and (possibly) continuous everywhere else.
%\smallskip
%\item This is exactly the same as the left censored and shifted random variable studied in Weeks 1-2.
%\end{itemize}
%}
%
%\frame{\frametitle{Density, CDF, SDF of the per-loss variable}
%\begin{itemize}
%\item We can write the density of $Y$ as
%\begin{equation*}
%f_{Y^L}(y) = \left\{
%\begin{array}{ll}
%F_X(d), & \text{for } y=0 \\
%f_X(y+d), & \text{for } y>0
%\end{array}
%\right. .
%\end{equation*}
%\item Its cumulative distribution function is
%\begin{equation*}
%F_{Y^L}(y) = F_X(y+d),
%\end{equation*}
%and its survival function is
%\begin{equation*}
%S_{Y^L}(y) = S_X(y+d).
%\end{equation*}
%\item Its hazard rate function is undefined at $y=0$.
%\end{itemize}
%}
%
%\section{Expectation}
%\subsection{per-loss variable}
%\frame{\frametitle{Expectation of the per-loss variable}
%\begin{itemize}
%\item The random variable $Y^L$ is often referred to as the \textit{claim amount paid per-loss event}.
%\smallskip
%\item Its expectation is
%\begin{equation*}
%\text{E}(Y) = \text{E}\left[(X-d)_+\right] = \int_d^{\infty} (x-d) f_X(x) dx.
%\end{equation*}
%\item Using integration by parts, we can show that this is equivalent to:
%\begin{equation*}
%\text{E}\left[(X-d)_+\right] = \int_d^{\infty} \left[1-F_X(x)\right] dx.
%\end{equation*}
%\item Because there are some losses that do not produce payments at all, it is clear that $\text{E}(Y) \leq \text{E}(X)$.
%\end{itemize}
%}
%
%\subsection{The case of the Normal}
%\frame{\frametitle{An example - Normal}
%Prove that if $X \sim \text{Normal}(\mu,\sigma^2)$, then the expected amount paid per loss event can be written as
%\begin{equation*}
%\text{E}\left[(X-d)_+\right] = \sigma \phi \left(\frac{d-\mu}{\sigma} \right) - (d-\mu) \left[1-\Phi\left(\frac{d-\mu}{\sigma}\right) \right].
%\end{equation*}
%\smallskip
%Here $\phi(\cdot)$ denotes the density of a standard Normal and $\Phi(\cdot)$ denotes the cumulative distribution function of a standard Normal.
%}
%
%\section{Ordinary deductible}
%\subsection{per-payment variable}
%\frame{\frametitle{Amount paid per-payment event}
%\begin{itemize}
%\item Define the random variable $Y^P$ to be the \textit{claim amount paid per payment event}:
%\begin{equation*}
%Y^P = Y^L|Y^L>0 = Y^L|X>d.
%\end{equation*}
%\item This is exactly the same as the excess loss random variable defined in Weeks 1-2.
%\smallskip
%\item Its expectation, sometimes called the \alert{mean excess loss}, is clearly larger than the expected amount paid per-loss event.
%\smallskip
%\item Expectation:
%\begin{equation*}
%\text{E}(Y^P) = \text{E}(Y^L|X>d) = \frac{\displaystyle \int_d^{\infty} (x-d) f_X(x) dx}{1-F_X(d)} = \frac{\text{E}(Y^L)}{1-F_X(d)}.
%\end{equation*}
%\end{itemize}
%}
%
%\frame{\frametitle{Density, CDF, SDF, hazard of the per-payment variable}
%\begin{itemize}
%\item Density function: $f_{Y^P}(y) = \dfrac{f_X(y+d)}{S_X(d)}$ for $y>0$
%\smallskip
%\item Cumulative distribution function: $F_{Y^P}(y) = \dfrac{F_X(y+d)-F_X(d)}{S_X(d)}$
%\smallskip
%\item Survival function: $S_{Y^P}(y) = \dfrac{S_X(y+d)}{S_X(d)}$
%\smallskip
%\item Hazard rate function: $h_{Y^P}(y) = \dfrac{f_X(y+d)}{S_X(y+d)}$.
%\end{itemize}
%}
%
%\subsection{The case of the Exponential}
%\frame{\frametitle{An example - Exponential}
%Suppose that the loss random variable $X \sim \text{Exp}(1/\lambda)$, i.e.
%\begin{equation*}
%f_X(x) = \lambda e^{-\lambda x}, \ \ \text{for } x>0.
%\end{equation*}
%
%\smallskip
%Derive expressions for $\text{E}\left[(X-d)_+\right]$ and the mean excess loss $\text{E}(Y^P)$.
%}
%
%\section{Franchise deductible}
%\frame{\frametitle{Franchise deductible}
%\begin{itemize}
%\item A \alert{franchise deductible} modifies the ordinary deductible by adding the deductible when there is a positive amount paid.
%\smallskip
%\item Once the loss $X$ exceeds the threshold $d$, the insurer pays the full loss $X$.
%\smallskip
%\item The \alert{per-loss} random variable for a franchise deductible is therefore
%\begin{equation*}
%Y^L = \left\{
%\begin{array}{ll}
%0, & \text{for } X \leq d \\
%X, & \text{for } X>d
%\end{array}
%\right. .
%\end{equation*}
%\item The \alert{per-payment} random variable for a franchise deductible is therefore $Y^P = X|X>d$.
%\smallskip
%\item The associated density, CDF, SDF and hazards for these random variables can be found on page 182 of Klugman, et al.
%\end{itemize}
%}

%\section{Expected costs}
%\subsection{ordinary/franchise deductibles}
%\frame{\frametitle{Expected costs}
%\begin{itemize}
%\item For an \alert{ordinary deductible}, the expected cost per loss is
%\begin{equation*}
%\text{E}(X) - \text{E}(X \wedge d)
%\end{equation*}
%and the expected cost per payment is
%\begin{equation*}
%\dfrac{\text{E}(X) - \text{E}(X \wedge d)}{1-F_X(d)}.
%\end{equation*}
%%\smallskip
%%\item For a \alert{franchise deductible}, the expected cost per loss is
%%\begin{equation*}
%%\text{E}(X) - \text{E}(X \wedge d) + d[1-F_X(d)]
%%\end{equation*}
%%and the expected cost per payment is
%%\begin{equation*}
%%\dfrac{\text{E}(X) - \text{E}(X \wedge d)}{1-F_X(d)} + d.
%%\end{equation*}
%\end{itemize}
%}
%
%\section{Policy limits}
%\frame{\frametitle{Policy limits}
%\begin{itemize}
%\item If the policy has a limit of say $u$, then the insurer is liable only to the extent it does not exceed $u$.
%\item The claim amount random variable is clearly $Y = \min(X,u)$ and can be formally written as
%\begin{equation*}
%Y = (X \wedge u) = \left\{
%\begin{array}{ll}
%X, & \text{for } X \leq u \\
%u, & \text{for } X>u
%\end{array}
%\right. .
%\end{equation*}
%\item Its density can be expressed as
%\begin{equation*}
%f_Y(y) = \left\{
%\begin{array}{ll}
%f_X(y), & \text{for } y < u \\
%1-F_X(u), & \text{for } y=u
%\end{array}
%\right. .
%\end{equation*}
%\item Its distribution function can be expressed as
%\begin{equation*}
%F_Y(y) = \left\{
%\begin{array}{ll}
%F_X(y), & \text{for } y < u \\
%1, & \text{for } y \geq u
%\end{array}
%\right. .
%\end{equation*}
%\end{itemize}
%}
%
%\subsection{expectation}
%\frame{\frametitle{Expectation with policy limit}
%Consider a policy with limit $u$. Show that the expected cost can be expressed as
%\begin{equation*}
%\text{E}(Y) = u - \int_0^u F_X(x)dx = \int_0^u \left[1-F_X(x)\right] dx.
%\end{equation*}
%Calculate this expected cost in the case where the loss $X$ has a Pareto distribution with $\alpha=3$ and $\theta=2,000$ for a coverage with policy limit of 3,000.
%}

\subsection{relationship with deductibles}
\frame{\frametitle{Relationships between deductibles and limits}
\begin{itemize}
\item For any loss random variable $X$, it can be shown that
\begin{equation*}
X = (X-d)_+ + (X \wedge d).
\end{equation*}
\item The interpretation to this is intuitively clear: for a policy with a deductible $d$, losses below $d$ are not covered and therefore can be covered by another policy with a limit of $d$.
\smallskip
\item The expectations are therefore equal:
\begin{equation*}
\text{E}(X) = \text{E}\left[(X-d)_+\right] + \text{E}(X \wedge d).
\end{equation*}
\item In the case where there is a deductible $d$, the insurer's savings can therefore be thought of as $S=(X \wedge d)$.
\item The expected savings (due to the deductible) expressed as a percentage of the loss (no deductible at all) is called the \alert{Loss Elimination Ratio}:
\begin{equation*}
\text{LER} = \frac{\text{E}(X \wedge d)}{\text{E}(X)}.
\end{equation*}
\end{itemize}
}

\section{Additional examples}
\subsection{Example 1}
\frame{\frametitle{Review Example}
An insurance company offers two types of policies: Type Q and Type R.

\smallskip
You are given:
\smallskip
\begin{itemize}
\item Type Q has no deductible, but has a policy limit of \$3,000.
\smallskip
\item Type R has no policy limit, but has a deductible of $d$.
\smallskip
\item Losses follow a Pareto$(\alpha,\theta)$ distribution with $\alpha = 3$ and $\theta = 2,000$.
\end{itemize}

\smallskip
Calculate $d$ so that both policies have the same expected claim amount per loss.
}



\section{Coinsurance}
\frame{\frametitle{Coinsurance}
\begin{itemize}
\item For policies with \alert{coinsurance}, claim amount is proportional to the loss amount by a coinsurance factor.
\smallskip
\item Coinsurance factor is denoted by $\alpha$, where $0<\alpha<1$ so that the claim payment random variable is
\begin{equation*}
Y = \alpha X.
\end{equation*}
\item Its density can be expressed as
\begin{equation*}
f_Y(y) = \frac{1}{\alpha}f_X \left(\frac{y}{\alpha}\right).
\end{equation*}
\item Its expected value is clearly $\text{E}(Y) = \alpha \text{E}(X)$.
\end{itemize}
}

\section{Combining policy modifications}
\frame{\frametitle{Combining policy modifications}
\begin{itemize}
\item It is possible to combine deductibles and/or policy limits together with coinsurance factors.
\smallskip
\item We shall adopt the convention that the coinsurance factor is applied after the application of any deductible or limit.
\smallskip
\item So for example, consider the case where we have a policy deductible $d$, a policy limit $u$ and a coinsurance factor $\alpha$. The amount paid per loss random variable is given by
\begin{equation*}
Y^L = \alpha \left[(X \wedge u) - (X \wedge d) \right].
\end{equation*}
\end{itemize}
}

\subsection{an illustration}
\frame{\frametitle{An illustration}
A Health Maintenance Organization (HMO) currently pays full cost of any emergency room care to its clients.

\smallskip
You are given that the cost of an emergency room care has an Exponential distribution with mean 1,000.

\smallskip
The company is evaluating the possible savings of imposing a deductible of \$200 per emergency room visit, to be paid by the client.

\smallskip
\begin{enumerate}
\item Calculate the resulting loss elimination ratio due to a deductible of \$200. Interpret this ratio.
\smallskip
\item Suppose the HMO decides to impose a per loss deductible of \$200 per emergency room visit, along with a policy limit of \$5,000 and a coinsurance factor of 80\%. For every visit to the emergency room, calculate the expected claim amount per loss event and the expected claim amount per payment event made by the HMO.
\end{enumerate}
}

\section{The effect of inflation}
\frame{\frametitle{The effect of inflation}
\begin{itemize}
\item Assume that there is a gap between the time of loss and the time the payments are made, and the insurer is obligated to cover inflation losses.
\smallskip
\item For modeling purposes, this means that instead of a loss of $X$, now the loss to consider is $(1+r)X$, assuming an inflation rate of $r$ during the period.
\smallskip
\item Note that for policies with deductibles, the deductible is subtracted after the inflation has been taken into account. Therefore, the effect on the payment amount is greater than $100r\%$ for two reasons:
\smallskip
\begin{enumerate}
\item There are now more claims exceeding the deductible.
\smallskip
\item The deductible amount is usually not increased for inflation, so that those claims exceeding the deductible will increase by more than the rate of inflation, even before the inflation.
\end{enumerate}
\end{itemize}
}

\subsection{calculating the expected claims}
\frame{\frametitle{Calculating the expected claim per loss}
\begin{itemize}
\item Assume a deductible of $d$, a coinsurance of $\alpha$, a policy limit of $u$, and the inflation rate of $r$.
\smallskip
\item The expected claim per loss will be
\begin{equation*}
\text{E}(Y^L) = \text{E}[\alpha((1+r)X \wedge u)] - \text{E}[\alpha((1+r)X \wedge d)].
\end{equation*}
\item This can be re-expressed as
\begin{equation*}
\text{E}(Y^L) = \alpha(1+r) \left[\text{E}\left(X \wedge \frac{u}{1+r}\right) - \text{E}\left( X \wedge \frac{d}{1+r} \right) \right].
\end{equation*}
\end{itemize}
}

\frame{\frametitle{Calculating the expected claim per payment}
\begin{itemize}
\item We can express the probability of the loss (after inflation) exceeding the deductible, and therefore producing a claim, as follows:
\begin{equation*}
\text{Pr}((1+r)X>d) = \text{Pr}\biggl(X>\frac{d}{1+r}\biggr) = 1- F_X\biggl(\frac{d}{1+r}\biggr).
\end{equation*}
\smallskip
\item The expected claim amount per payment is therefore
\begin{equation*}
\text{E}(Y^P) = \frac{\text{E}(Y^L)}{1- F_X\left(\frac{d}{1+r}\right)}.
\end{equation*}
\end{itemize}
}

\subsection{illustration}
\frame{\frametitle{Illustration}
To illustrate, consider the case of the HMO in the previous example with a \$200 deductible, \$5,000 policy limit and an 80\% coinsurance factor.

\smallskip
Now, assume a 5\% uniform inflation. Calculate the new expected claim amounts per loss and per payment.
}

\section{Deductibles in Aggregate Loss Models}
\subsection{Individual vs Aggregate Deductibles}
\frame{\frametitle{Individual vs Aggregate Deductibles}
\begin{itemize}
    \item In the context of insurance, especially when dealing with aggregate losses, understanding the distinction between individual and aggregate deductibles is crucial.
    \medskip
    \item \textbf{Individual Deductibles (Per Loss Deductible):}
    \begin{itemize}
        \item Applied to each individual loss.
        \item The insurer pays only the part of each claim that exceeds the deductible.
        \item Common in standard insurance policies.
    \end{itemize}
    \medskip
    \item \textbf{Aggregate Deductibles (Stop Loss Deductible):}
    \begin{itemize}
        \item Applied to the total amount of losses over a specified period.
        \item The insurer starts paying for losses only when the total aggregate losses exceed the deductible amount.
        \item Used in policies with high frequency of small losses.
    \end{itemize}
    \medskip
    \item These deductibles serve different purposes:
    \begin{itemize}
        \item \textit{Individual Deductibles} manage the frequency of small claims.
        \item \textit{Aggregate Deductibles} manage the total cost of claims over a period.
    \end{itemize}
\end{itemize}
}

\section{Aggregate Loss Scenario Example}
\subsection{Poisson Number of Claims and Exponential Severity}
\frame{\frametitle{Aggregate Loss with Poisson Claims and Exponential Severity}
\begin{itemize}
        \item Number of claims ($N$) follows a Poisson distribution.
        \item Severity of each claim ($X_i$) follows an Exponential distribution with $\theta = 750$.
        \item Individual deductible: $d = 250$ per claim.

 \end{itemize}
 Calculate the expected aggregate loss $E(S)$ both with and without the deductible.
}

\section{Net Stop-Loss Premium}
\frame{\frametitle{Net Stop-Loss Premium}
\begin{itemize}
    \item The net stop-loss premium is an important concept in actuarial science, used to determine the premium for reinsurance contracts.
    \medskip
    \item It is defined as the expected payment under a stop-loss reinsurance with a deductible $d$, and can be calculated as:
    \begin{equation*}
        E(S-d)_+ = \int_d^\infty [1-F_S(x)]dx = \int_d^\infty (x-d)f_S(x)dx,
    \end{equation*}
    where $S$ is the aggregate loss, $F_S(x)$ is the cumulative distribution function of $S$, and $f_S(x)$ is the probability density function of $S$.
    \medskip
    \item For discrete distributions, this becomes:
    \begin{equation*}
        E(S-d)_+ = \sum_{x=d+1}^{\infty} (x-d)f_S(x).
    \end{equation*}
\end{itemize}
}

\section{Example: Calculation of Net Stop-Loss Premium}
\frame{\frametitle{Example: Setup}
\begin{itemize}
    \item Consider a portfolio where the number of claims $N$ and the amount of each claim $X$ are both discrete random variables.
    \item Let $N$ have the following distribution:
    \begin{itemize}
        \item $\text{Pr}(N = 0) = 0.6$
        \item $\text{Pr}(N = 1) = 0.4$
    \end{itemize}
    \item Let $X$ have the following distribution:
    \begin{itemize}
        \item $\text{Pr}(X = 100) = 0.5$
        \item $\text{Pr}(X = 200) = 0.5$
    \end{itemize}
    \item Our goal is to calculate $\text{Pr}(S = x)$ for all possible values of $x$ and determine the net stop-loss premium for a deductible $d$.
\end{itemize}
}
\subsection{Probabilities for Aggregate Loss $S$}
\frame{\frametitle{Probabilities for Aggregate Loss $S$}
\begin{itemize}
    \item Based on the distributions of $N$ and $X$, the aggregate loss $S$ can take a few discrete values.
    \item The table below shows the probabilities $\text{Pr}(S = x)$:
\end{itemize}

\begin{table}[h]
\centering
\begin{tabular}{cc}
\hline
Aggregate Loss $S$ & Probability $\text{Pr}(S = x)$ \\ \hline
0 & 0.6 \\
100 & 0.2 \\
200 & 0.2 \\ \hline
\end{tabular}
\end{table}

\begin{itemize}
    \item These probabilities are calculated based on the combined probabilities of $N$ and $X$.
\end{itemize}
}


\subsection{Net Stop-Loss Premium Calculation}
\frame{\frametitle{Net Stop-Loss Premium Calculation}
\begin{itemize}
    \item Given a deductible $d = 50$, we calculate the net stop-loss premium.
    \item The net stop-loss premium formula is:
    \begin{equation*}
        E(S-d)_+ = \sum_{x=d}^{\infty} (x-d)f_S(x).
    \end{equation*}
    \item Using the probabilities from the previous table:
\end{itemize}

\begin{equation*}
\begin{aligned}
E(S-50)_+ &= (100-50) \times 0.2 + (200-50) \times 0.2 \\
          &= 50 \times 0.2 + 150 \times 0.2 \\
          &= 10 + 30 \\
          &= 40.
\end{aligned}
\end{equation*}

\begin{itemize}
    \item The net stop-loss premium with a deductible of $50$ is $40$.
\end{itemize}
}

\section{Recursive Relationships for Net Stop-Loss Premium}
\subsection{Rule 1: Conditional Expectation Formula}
\frame{\frametitle{Rule 1: Conditional Expectation Formula}
\begin{itemize}
    \item When the probability of the aggregate loss $S$ falling between two values $a$ and $b$ is zero (i.e., $\text{Pr}(a < S < b) = 0$), the net stop-loss premium can be calculated using a linear interpolation:
    \begin{equation*}
        E[(S-d)_+] = \frac{b-d}{b-a}E[(S-a)_+] + \frac{d-a}{b-a}E[(S-b)_+]
    \end{equation*}
    \item This relationship is particularly useful for discrete loss distributions where $S$ can only take specific values.
\end{itemize}
}

\subsection{Example for Rule 1}
\frame{\frametitle{Example for Rule 1}
\begin{itemize}
    \item Consider an aggregate loss $S$ that can only take values 100, 500, or 1000.
    \item Let's calculate the net stop-loss premium for a deductible $d = 300$, using $a = 100$ and $b = 500$.
    \item Assume $E[(S-100)_+] = 400$ and $E[(S-500)_+] = 750$.
    \item Applying the formula:
    \begin{equation*}
        E[(S-300)_+] = \frac{500-300}{500-100} \times 400 + \frac{300-100}{500-100} \times 750
    \end{equation*}
    \item This yields $E[(S-300)_+] = 575$.
\end{itemize}
}

\subsection{Rule 2: Discrete Loss Increments}
\frame{\frametitle{Rule 2: Discrete Loss Increments}
\begin{itemize}
    \item This recursive formula applies when the aggregate loss $S$ takes values in multiples of a fixed amount $h$.
    \item The formula is given by:
    \begin{equation*}
        E[(S-(j+1)h)_+] = E[(S-jh)_+] - h(1 - F_S(jh))
    \end{equation*}
    \item It's used to calculate the expected excess loss over each successive deductible level $jh$.
\end{itemize}
}

\subsection{Example for Rule 2}
\frame{\frametitle{Example for Rule 2}
\begin{itemize}
    \item Suppose $S$ takes values in multiples of $h = 100$ and $E[(S-100)_+] = 200$.
    \item Let's calculate $E[(S-200)_+]$ and $E[(S-300)_+]$.
    \item Assume $\text{Pr}(S \geq 100) = 0.8$ and $\text{Pr}(S \geq 200) = 0.6$.
    \item Applying the formula for $E[(S-200)_+]$:
    \begin{equation*}
        E[(S-200)_+] = 200 - 100 \times  0.8 = 120
    \end{equation*}
    \item Similarly, for $E[(S-300)_+]$:
    \begin{equation*}
        E[(S-300)_+] = E[(S-200)_+] - 100 \times 0.6 = 60
    \end{equation*}
\end{itemize}
}

\subsection{Practice Problem}
\frame{\frametitle{Practice Problem Using Recursive Formulas}
\begin{itemize}
    \item You are given an insurance portfolio where the aggregate loss $S$ can only take values in multiples of $100$. The probabilities for $S$ are as follows:
    \begin{itemize}
        \item $\text{Pr}(S = 100) = 0.1$
        \item $\text{Pr}(S = 200) = 0.15$
        \item $\text{Pr}(S = 300) = 0.25$
       
       $\vdots$
    \end{itemize}
    \item  You know that $E[(S-100)_+] = 460$. Using the second recursive formula, calculate the net stop-loss premium for a deductible of \$300.
\end{itemize}
}






\end{document}
