\documentclass[compress,mathserif]{beamer}

\usepackage{amssymb}
\usepackage{amsfonts}
\usepackage{amsmath}
\usepackage{amsxtra}
\usepackage{hyperref}
\usepackage{tensor}
\usepackage{dsfont}

\newcommand{\ntp}{\\ \vspace{1em}}

\mode<presentation> {
  \usetheme{default}
  \useoutertheme{infolines}
  \setbeamercovered{transparent}
  \beamertemplateballitem
}

\title{Ratemaking}
\author{}
\institute[Stat 346]{Stat 346 - Short-term Actuarial Math}
\date[ BYU]{}
\subject{Stat 346}

\begin{document}

\begin{frame}
 \titlepage
\end{frame}

\begin{frame}{Example 4.3 from book}
Calculate the indicated average rate given the following information. 
\begin{itemize}
\item Expected effective losses - 30,000,000
\item Earned Exposure Units - 1,000,000
\item Earned Premium at current rates - 45,000,000
\item Current average rate - 45
\item Fixed Expenses - 5,000,000
\item Fixed expenses per exposure unit - 5
\item Premissable Loss Ratio - 0.75
\end{itemize}

\end{frame}

\begin{frame}{Example 4.3 from book}
\textbf{Loss Cost method}
\[\text{Expected Effective Loss Cost} = \frac{30,000,000}{1,000,000} = 30\]
\[\text{New rate} = \frac{30 + 5}{0.75} = 46.67\]

\textbf{Loss ratio method}
\[\text{Expected Effective loss ratio} =  \frac{30,000,000}{45,000,000} = 0.66\]
\[\text{Fixed Expense Ratio} = \frac{5}{45} = 0.1111\]
\[\text{Indicated rate change} = \frac{0.666 + 0.111}{0.75} = 1.037037\]
So the new rate would be the current rate times the rate change, 
\[\text{New Rate} = 45(1.037) = 46.67\]
\end{frame}


\begin{frame}{Example}
\small
Full Loss Cost problem. You know that reported claims for the last few year follow this claims triangle. 
\begin{table}[htbp]
    \centering
    \begin{tabular}{|c|c|c|c|}
    \hline
    \multicolumn{4}{|c|}{\textbf{Claims Triangle}} \\ \hline
    \textbf{Accident Year} & \textbf{DY0} & \textbf{DY1} & \textbf{DY2} \\ \hline
    \textbf{2022} & $150,000$ & $200,000$ & $250,000$ \\ \hline
    \textbf{2023} & $180,000$ & $220,000$ &  \\ \hline
    \textbf{2024} & $200,000$ & &  \\ \hline
    \end{tabular}
\end{table}
\small
\begin{itemize}
\item The permissible loss ratio is 75\%. 
\item The trend factor for exponential claims growth is $\delta = 0.075$. 
\item The first two loss development factors for losses is estimated to be 1.27 and 1.25.
\item There are 1,000 exposure units and fixed expenses per exposure is \$30. 
\item Assume no additional losses from years prior to 2022 and no tail factor
\end{itemize}
What should the rate be based on the loss cost method for a new one year policy starting in 2025? Use a weighted average of 70\% 2024 losses and 30\% 2023 losses. 

\end{frame}

\begin{frame}{Example}
First off, losses must be developed. We'll divide by the exposure unit of 1,000 right off. 
\begin{itemize}
\item \textbf{2022}: $250$
\item \textbf{2023}: $220(1.25) = 275$ per exposure
\item \textbf{2024}: $200 (1.27)(1.25) = 317.5$ per exposure
\end{itemize} 
We need to trend this for 2025 policies. Projected losses from 2024 are \[317.5 \times \exp{(.075 (1.5))} = 355.30\] The time period is 1.5 using parallelograms. Projected losses for 2023 are  \[279.4 \times \exp{(.075 (2.5))} = 337.02\]. We will use losses of $.7(355.30) + .3(337.02) = 349.82$. Then the rate is \[(349.82+30)/.75 = 506.43\]
\end{frame}

\begin{frame}{Example}
Full Loss Ratio problem. Let's borrow the information we got from the last problem. 
\begin{itemize}
\item Expected effective losses are \$349.82 per exposure unit
\item  Fixed expenses per exposure are \$30.  
\item PLR is 0.75
\end{itemize} 
In 2022 there was \$460 of earned exposure. We want to convert this to rates as of January 1st, 2025. There was a 5\% rate increase on April 1st, 2022 and a 6\% rate change on November 1, 2023. What are 2022 earned exposure as of January 1st, 2025? Once you find it, use that to find the rate change for 2025. 

\end{frame}

\begin{frame}{Example}
Total rate change to 2025 is $(1.05)(1.06) = 1.113$. The rate level for 2022 using parallelogram is $(0.7185 \times 1) + (0.2815 \times 1.05) = 1.014$. The on level factor then is $\frac{1.113}{1.014} = 1.0976$. So earned premiums per exposure is $460 \times 1.0976 = 504.87$. 

Now we have that we can find expected effective loss ratio is $349.82 / 504.87  = 0.693$. The fixed expense ratio is $30/504.87 = 0.0594$. Then the indicated rate change is \[\frac{0.693 + 0.0594}{0.75} = 1.003\]
This says rates should increase by 0.3\%. If the current rates were \$500 then the new rates would be $(500)(1.003) = 501.53$. 

\end{frame}




\end{document}
