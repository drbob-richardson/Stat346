\documentclass[compress,mathserif]{beamer}

\usepackage{amssymb}
\usepackage{amsfonts}
\usepackage{amsmath}
\usepackage{amsxtra}
\usepackage{hyperref}
\usepackage{tensor}
\usepackage{dsfont}

\mode<presentation> {
  \usetheme{default}
  \useoutertheme{infolines}
  \setbeamercovered{transparent}
  \beamertemplateballitem
}

\title[Creating New Distributions]{Creating New Distributions}
\author{Section 5.2}
\institute[Stat 346]{Stat 346 - Short-term Actuarial Math}
\date[Brian Hartman - BYU]{}
\subject{Stat 477}

\begin{document}

\begin{frame}
 \titlepage
\end{frame}

\section{Generating new distributions}
\frame{\frametitle{Some methods to generate new distributions}
\begin{itemize}
\item There are many methods to generate new distributions; some of these methods allow us to give in-depth interpretation to the distributions.
\smallskip
\item Among the methods used can be sub-divided into:
\smallskip
\begin{enumerate}
\item Addition of several random variables
\smallskip
\begin{itemize}
\item For example, sums of (independent) Exponentials give a Gamma. This method will not be further explored.
\end{itemize}
\smallskip
\item Transformation of random variables
\smallskip
\begin{itemize}
\item scalar multiplication
\smallskip
\item power operations
\smallskip
\item exponentiation (or logarithmic transformation)
\end{itemize}
\smallskip
\item Mixing of distributions
\smallskip
\begin{itemize}
\item frailty models
\end{itemize}
\smallskip
\item Spliced distributions
\end{enumerate}
\smallskip
\item Section 5.2 of Klugman, et al.
\end{itemize}
}

\subsection{general theory}
\frame{\frametitle{The general theory of transformation}
Suppose we are interested in deriving the distribution of $Y=g(X)$, where $X$ has a known distribution function. Assume that the function $g$ is a one-to-one transformation (i.e. invertible).
\begin{itemize}
\item It can be shown that the distribution function of $Y$ can be expressed as
\begin{equation*}
F_Y(y) = F_X(g^{-1}(y)),
\end{equation*}
in the case of increasing transformation. If decreasing, we have
\begin{equation*}
F_Y(y) = 1-F_X(g^{-1}(y)).
\end{equation*}
\item Its density can be explicitly written as
\begin{equation*}
f_Y(y) = f_X(g^{-1}(y)) \times \left|\frac{dg^{-1}(y)}{dy}\right|.
\end{equation*}
\end{itemize}

}

\section{Some examples}
\subsection{scalar transformations}
\frame{\frametitle{Scalar transformations}
\begin{itemize}
\item In the case where $Y=aX$ for some $a>0$, then this is called a \alert{scalar} transformation and its density function can be expressed as
\begin{equation*}
f_Y(y) = \frac{1}{a}f_X(y/a).
\end{equation*}
\item Insurance interpretation: if $X$ denotes claims, then scalar transformation can be interpreted as applying inflation factor across all levels of claims.
\smallskip
\item A family of distributions that is closed under scalar multiplication (i.e. after scalar transformation, the new random variable remains in the same family) is called a \alert{scale family of distributions}.
\smallskip
\item Some scale families are:
\begin{itemize}
\item Normal
\smallskip
\item Exponential (Example 5.1)
\smallskip
\item Pareto
\end{itemize}
\end{itemize}
}

\subsection{power transformations}
\frame{\frametitle{Power transformations}
A \alert{power} transformation involves raising the random variable by a power such as
\begin{equation*}
Y = X^{1/\tau} \ \ \text{or} \ \ Y=X^{-1/\tau},
\end{equation*}
where $\tau>0$.

\medskip
In the first case, we have a \alert{transformed} $X$ distribution; the other case, we have an \alert{inverse transformed} $X$ distribution.

\medskip
In the special case where $Y=X^{-1}$, we have an \alert{inverse} $X$ distribution.
}

\frame{\frametitle{Distribution and density functions of power transformations}
It is easy to show the following results:
\smallskip
\begin{itemize}
\item In the transformed case where $Y=X^{1/\tau}$, we have
\begin{equation*}
F_Y(y) = F_X(y^\tau) \ \ \text{and} \ \ f_Y(y) = \tau y^{\tau-1} f_X(y^{\tau}).
\end{equation*}
\item In the inverse transformed case where $Y=X^{-1/\tau}$, we have
\begin{equation*}
F_Y(y) = 1-F_X(y^{-\tau}) \ \ \text{and} \ \ f_Y(y) = \tau y^{-\tau-1} f_X(y^{-\tau}).
\end{equation*}
\item In the inverse case where $Y=X^{-1}$, we have
\begin{equation*}
F_Y(y) = 1-F_X(y^{-1}) \ \ \text{and} \ \ f_Y(y) = \frac{1}{y^2} f_X(1/y).
\end{equation*}
\end{itemize}
}

\frame{\frametitle{Example 5.2}
Let $X$ be exponentially distributed with mean parameter 1.

\medskip
Derive the cumulative distribution and density functions of the transformed, inverse transformed and inverse random variables.

\medskip
Note that we derive:
\begin{itemize}
\item \alert{Inverse Exponential} distribution: for the case of the inverse distribution after a scale transformation
\smallskip
\item \alert{Weibull} distribution: for the case of the transformed distribution after a scale transformation
\smallskip
\item \alert{Inverse Weibull} distribution: for the case of the inverse transformed distribution after a scale transformation
\end{itemize}
}

\frame{\frametitle{Illustrative example 1}
Suppose $X$ is a random variable with cumulative distribution function
\begin{equation*}
F_X(x) = 1 - (1+x^c)^{-\gamma}, \ \ x>0, \ \ c>0, \ \ \gamma>0.
\end{equation*}

%\begin{enumerate}
\medskip
Derive the probability density function of the inverse of $X$, i.e. $Y=1/X$.

%\medskip
%\item Derive an explicit form of the higher moments of $Y$: $\text{E}(Y^k)$, for $k=1,2,\ldots$.
%\end{enumerate}
}

\frame{\frametitle{Illustrative example 2}
Suppose $X$ is an exponential random variable with mean parameter equal to 1.

\medskip

%\begin{enumerate}
Derive the distribution of $Y=\alpha X^{1/\beta}$, for $\alpha>0,\beta>0$. Specify its density function. The distribution of $Y$ is called a Weibull and we can write $Y \sim \text{Weibull}(\alpha,\beta)$, where $\alpha$ is obviously a scale parameter.
%
%\medskip
%\item Derive the mean and variance of $Y$.
%
%\medskip
%\item Assume that the future lifetime (in years) of an iPhone follows a Weibull distribution with mean 8 and variance 320. Calculate the probability that an iPhone fails within the first 5 years of use.
%\end{enumerate}
}

\subsection{exponentiation}
\frame{\frametitle{Exponentiation}
Suppose $Y=\exp(X)$. For $y>0$, its distribution function can be expressed as
\begin{equation*}
F_Y(y) = F_X(\log y)
\end{equation*}
and its corresponding density as
\begin{equation*}
f_Y(y) = \frac{1}{y}f_X(\log y)
\end{equation*}
Derive the distribution/density functions corresponding to the exponential transformation $Y=\exp(X)$ when:
\begin{itemize}
\item $X \sim \text{N}(\mu,\sigma^2)$
\smallskip
\item $X \sim \text{Exp}(1)$
\end{itemize}
}

\section{Mixtures of distributions}
\frame{\frametitle{Mixtures of distributions}
A random variable $X$ is said to be a \alert{mixture} of distributions if its distribution function has one of the following forms:
\begin{enumerate}
\item \alert{Discrete} mixture: $F_X(x) = \sum_i a_i F_{X_i}(x)$, for some sequence of random variables $X_1,X_2,\ldots$ and some sequence of positive numbers $a_1,a_2,\ldots$ satisfying $\sum_i a_i =1$.
\smallskip
\item \alert{Continuous} mixture: $F_X(x) = \int_{\infty}^{-\infty} F_{X|\Lambda=\lambda}(x) f_\Lambda(\lambda) d\lambda$, for some random variable $\Lambda$ satisfying $\int_{\infty}^{-\infty} f_\Lambda(\lambda) d\lambda =1$.
\end{enumerate}
\smallskip
In terms of actuarial/insurance applications:
\smallskip
\begin{itemize}
\item Discrete mixtures arise in situations where the risk class of a policyholder is uncertain, and the number of possible risk classes is discrete.
\smallskip
\item Continuous mixtures arise when a risk parameter from the loss distribution is uncertain and the uncertain parameter is continuous.
\end{itemize}
}

\subsection{examples}
\frame{\frametitle{Illustrative example 1}
An insurer has two groups of policyholders: the good and the bad risks. The insurer has a portfolio where 75\% are considered good risks.

\medskip
The claim distributions for both groups of risks are Exponential. The average claim amount of a good risk policyholder is \$100 while for bad risk, it is twice that.

\medskip
A new customer whose risk class is not known with certainty, has just recently purchased a policy from the insurer.

\medskip
Calculate the probability that this new customer will claim an amount exceeding \$150.
}

\frame{\frametitle{Illustrative example 2}
Consider a claims random variable $X$ that, given a risk classification (random) parameter $\Lambda$, can be modeled as an Exponential random variable with
\begin{equation*}
P(X\leq x| \Lambda=\lambda) = 1 - e^{-\lambda x}, \ \ \text{for } x>0.
\end{equation*}

\smallskip
Assume that $\Lambda$ has a Gamma$(\alpha,1/\theta)$ distribution.

\medskip

Show that the unconditional distribution of $X$ is a Pareto.


}

\subsection{mean and variance of mixtures}
\frame{\frametitle{Deriving the mean and variance of mixtures}
Suppose that $X$ is a mixture with mixing variable $\Lambda$. Then the unconditional mean and variance can be determined using the following formulas:

\medskip
\begin{itemize}
\item \alert{Law of iterated expectations}
\begin{equation*}
\text{E}(X) = \text{E}_\Lambda [\text{E}(X|\Lambda)],
\end{equation*}
or in general, we have
\begin{equation*}
\text{E}(X^k) = \text{E}_\Lambda [\text{E}(X^k|\Lambda)].
\end{equation*}
\item \alert{Conditional variance formula}
\begin{equation*}
\text{Var}(X) = \text{E}_\Lambda [\text{Var}(X|\Lambda)] + \text{Var}_\Lambda [\text{E}(X|\Lambda)].
\end{equation*}
\end{itemize}
}

\frame{\frametitle{Illustration}
The policyholders of an insurance company fall into one of two classes. The claims distributions for each class are given in the following table:

\smallskip
\begin{center}
\begin{tabular}{ccccccc}
\cline{1-3}\cline{5-7}
\multicolumn{3}{c}{Class 1} &  & \multicolumn{3}{c}{Class 2} \\ \cline{1-3}\cline{5-7}
claim size &  & probability &  & claim size &  & probability \\ \hline
1,000 &  & 0.20 &  & 1,000 &  & 0.70 \\
5,000 &  & 0.50 &  & 5,000 &  & 0.20 \\
10,000 &  & 0.30 &  & 10,000 &  & 0.10 \\ \hline
\end{tabular}
\end{center}

\smallskip
There are 30\% of policyholders in class 1, while the remaining policyholders are in class 2. Denote by $L_1$ the claim incurred by a randomly selected policyholder from class 1, while $L_2$ from class 2.

\smallskip
Let $L$ denote the claim incurred by a randomly selected policyholder whose risk class is unknown.

\smallskip
Calculate the mean and variance of $L$.
}

%\frame{\frametitle{Textbook examples}
%Review the following examples from the book:
%\medskip
%\begin{itemize}
%\item Example 5.5
%\smallskip
%\item Example 5.6
%\end{itemize}
%}

%\section{Frailty models}
%\frame{\frametitle{Frailty models}
%Define $\Lambda$ to be a positive, \alert{frailty} random variable such that, conditional on $\Lambda=\lambda$, the hazard rate of $X$ is of the form
%\begin{equation*}
%h_{X|\Lambda}(x|\lambda) = \lambda a(x),
%\end{equation*}
%for some specified and known function $a(x)$ of $x$.
%
%\medskip
%It can be shown that the conditional survival function of $X|\Lambda$ is
%\begin{equation*}
%S_{X|\Lambda}(x|\lambda) = \exp \biggl(-\int_0^x h_{X|\Lambda}(z|\lambda) dz \biggr) = e^{-\lambda A(x)}
%\end{equation*}
%where $A(x) = \int_0^x a(z) dz$.
%
%\medskip
%The (unconditional) survival function can then be derived as
%\begin{equation*}
%S_X(x) = \text{E} \left(e^{-\lambda A(x)}\right) = M_\Lambda (-A(x)),
%\end{equation*}
%where $M_\Lambda$ is the mgf of the frailty. %Consider: Example 5.7
%}

%\section{Spliced distributions}
%\frame{\frametitle{Spliced distributions}
%\begin{itemize}
%\item A \alert{spliced} distribution is one whose form of distribution is different in different portions of the domain of the random variable.
%\smallskip
%\item An interpretation in insurance claims is that the distributions vary by size of claims.
%\smallskip
%\item To illustrate, consider a two-spliced distribution:
%\begin{equation*}
%f_X(x) = \left\{
%\begin{array}{ll}
%p_1 \cdot f_1(x), & \text{for } 0<x<c \\
%p_2 \cdot f_2(x), & \text{for } c \leq x < \infty
%\end{array}
%\right. .
%\end{equation*}
%where $p_1+p_2 = 1$ and $f_1$ and $f_2$ are both legitimate density functions on the corresponding intervals.
%\smallskip
%\item This concept can be extended to a \alert{$k$-component spliced distributions}.
%%\item Review Example 5.9 for an illustration.
%\end{itemize}
%}
%
%\subsection{SOA question}
%\frame{\frametitle{SOA question}
%An actuary for a medical device manufacturer initially models the failure time for a particular device with an Exponential distribution with mean 4 years.
%
%\medskip
%This distribution is replaced with a spliced model whose density function:
%\begin{itemize}
%\item is Uniform over [0,3]
%\smallskip
%\item is proportional to the initial modeled density function after 3 years
%\smallskip
%\item is continuous
%\end{itemize}
%
%\smallskip
%Calculate the probability of failure in the first 3 years under the revised distribution.
%}

\end{document}
