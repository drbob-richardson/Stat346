\documentclass{beamer}
\usepackage{beamerinnerthemecircles, beamerouterthemeshadow}
\usepackage{amsmath, amsfonts, amscd, epsfig, amssymb, amsthm}
\usepackage{amsmath, amssymb, amsfonts, amsthm, epsfig, graphicx,float,enumerate}
\usepackage{graphicx,float,enumerate,natbib, subfigure, caption,dsfont}
\usepackage{graphicx,float,enumerate,relsize,bm,subfigure}
\usepackage[gen]{eurosym}
\usepackage{siunitx}

\DeclareSIUnit{\pers}{pers}
\DeclareSIUnit{\EUR}{\text{\euro}}
\sisetup{
  per-mode = fraction,
  inter-unit-product = \ensuremath{{}\cdot{}},
}

\usepackage{tikz}
\usetikzlibrary{matrix}




\mode<presentation> {
    \usetheme{Frankfurt} %Bergen, Berkely, Berlin, Boadilla, CambridgeUS, Darmstadt,
                          %Frankfurt, Goettingen, Singapore, Warsaw
    \usecolortheme{default} %beetle, seahorse, wolverine, dolphin, beaver
}

\title{Two More Option Things}

\author{}
\date{}
\usepackage{lipsum}

\newcommand\Wider[2][3em]{%
\makebox[\linewidth][c]{%
  \begin{minipage}{\dimexpr\textwidth+#1\relax}
  \raggedright#2
  \end{minipage}%
  }%
}
% If you want to display your table of contents at every Section/Subsection:
%\AtBeginSection[] {
%\begin{frame}<beamer>{Table of Contents}
%\tableofcontents[currentsection,currentsubsection]
%\end{frame}
%}

\begin{document}

\maketitle
\begin{frame}
\frametitle{Put-Call Parity}

\textbf{Definition:} Put-Call Parity establishes a relationship between the prices of a European call option and a European put option with the same strike price and expiration date.

\textbf{Formula:}
\[ C - P = S - K e^{-rT} \]
where:
\begin{itemize}
  \item $C$ = price of the call option
  \item $P$ = price of the put option
  \item $S$ = current stock price
  \item $K$ = strike price
  \item $r$ = risk-free interest rate
  \item $T$ = time to maturity
\end{itemize}

\end{frame}

\begin{frame}
\frametitle{Put-Call Parity Example}

\textbf{Given:}
\begin{itemize}
  \item Stock price $(S) = \$100$
  \item Strike price $(K) = \$100$
  \item Risk-free rate $(r) = 5\%$
  \item Time to maturity $(T) = 1$ year
  \item Call option price $(C) = \$10$
\end{itemize}

\textbf{Find:} Price of the put option $(P)$

\textbf{Calculation:}
\[ P = C - S + K e^{-rT} = 10 - 100 + 100 e^{-0.05 \times 1} \]
\[ P = 10 - 100 + 95.12 = \$5.12 \]

\end{frame}

\begin{frame}
\frametitle{Put-Call Parity Practice}

\textbf{Practice Problem:}
Calculate the price of the put option given:
\begin{itemize}
  \item Stock price $(S) = \$150$
  \item Strike price $(K) = \$155$
  \item Risk-free rate $(r) = 3\%$
  \item Time to maturity $(T) = 6$ months
  \item Call option price $(C) = \$8$
\end{itemize}
\end{frame}

\begin{frame}
\frametitle{Delta Hedging}

\textbf{Definition:} Delta hedging is an options strategy that aims to reduce, or hedge, the directional risk associated with price movements in the underlying asset by adjusting the position in the underlying asset and its options.

\textbf{Delta ($\Delta$):}
\[ \Delta = \frac{\partial C}{\partial S} \]
- For calls, $\Delta$ ranges from 0 to 1.

- For puts, $\Delta$ ranges from -1 to 0.

\end{frame}

\begin{frame}
\frametitle{Delta Hedging for Calls}
\framesubtitle{Strategy and Calculations}

\textbf{Black-Scholes Formula for $\Delta$ (Call Option):}
\[ \Delta_C = N(d_1) \]
where \( d_1 = \frac{\log(S/K) + (r + 0.5 \sigma^2) T}{\sigma \sqrt{T}} \)

\textbf{Strategy:}
\begin{itemize}
  \item Short $\Delta_C \times \text{Number of Options}$ shares of the underlying stock.
  \item Shorting shares means selling shares you do not currently own, expecting to buy them back at a lower price.
\end{itemize}

\textbf{Purpose:}
This strategy ensures that gains in the option's value due to increases in the underlying stock's price are offset by losses in the shorted stock position, leading to a less volatile overall investment position.

\end{frame}

\begin{frame}
\frametitle{Delta Hedging for Puts}
\framesubtitle{Strategy and Calculations}

\textbf{Black-Scholes Formula for $\Delta$ (Put Option):}
\[ \Delta_P = -N(-d_1) = N(d_1) - 1 \]
where \( d_1 = \frac{\log(S/K) + (r + 0.5 \sigma^2) T}{\sigma \sqrt{T}} \)

\textbf{Strategy:}
\begin{itemize}
  \item Buy $\left| \Delta_P \right| \times \text{Number of Options}$ shares of the underlying stock.
  \item Buying shares (longing) involves purchasing shares with the expectation that their value will increase.
\end{itemize}

\textbf{Purpose:}
Longing shares in the context of put options hedging ensures that losses due to a decrease in the stock price (which increases the value of the put option) are offset by gains in the longed stock position, stabilizing the overall investment value.

\end{frame}

\begin{frame}
\frametitle{Understanding Shorting and Longing in Hedging}

\textbf{Shorting Shares:}
\begin{itemize}
  \item Investors sell shares they do not own by borrowing them.
  \item The goal is to buy back the shares at a lower price and return them to the lender, pocketing the difference as profit.
\end{itemize}

\textbf{Longing Shares:}
\begin{itemize}
  \item Investors purchase shares outright with the belief that the share price will increase.
  \item Profits are made when the shares are sold at a higher price than they were bought.
\end{itemize}

\textbf{Role in Hedging:}
Both strategies are used to counterbalance the directional risk associated with holding options, aiming to neutralize the financial impact of significant price swings in the underlying asset.

\end{frame}


\begin{frame}
\frametitle{Delta Hedging Example: Call}

\textbf{Given:}
\begin{itemize}
  \item Stock price $(S) = \$100$
  \item Delta of call option $(\Delta_C) = 0.6$
  \item Number of options = 100
\end{itemize}

\textbf{Objective:} Construct a delta-neutral portfolio

\textbf{Action:} Short 60 shares of the stock (since $100 \times 0.6 = 60$)

\end{frame}

\begin{frame}
\frametitle{Delta Hedging Example: Put}

\textbf{Given:}
\begin{itemize}
  \item Stock price $(S) = \$100$
  \item Delta of put option $(\Delta_P) = -0.4$
  \item Number of options = 100
\end{itemize}

\textbf{Objective:} Construct a delta-neutral portfolio

\textbf{Action:} Buy 40 shares of the stock (since $100 \times -0.4 = -40$, and we negate the negative sign by buying)

\end{frame}

\begin{frame}
\frametitle{Delta Hedging Practice}

\textbf{Practice Problems:}
\begin{enumerate}
  \item With a delta of 0.5 for a call option, how many shares should be shorted for a delta-neutral position if you own 150 options?
  \item With a delta of -0.3 for a put option, how many shares should be bought for a delta-neutral position if you own 200 options?
\end{enumerate}
\end{frame}


\end{document}







