\documentclass{beamer}
\usepackage{beamerinnerthemecircles, beamerouterthemeshadow}
\usepackage{amsmath, amsfonts, amscd, epsfig, amssymb, amsthm}
\usepackage{amsmath, amssymb, amsfonts, amsthm, epsfig, graphicx,float,enumerate}
\usepackage{graphicx,float,enumerate,natbib, subfigure, caption,dsfont}
\usepackage{graphicx,float,enumerate,relsize,bm,subfigure}

\usepackage{tikz}
\usetikzlibrary{matrix}

\theoremstyle{plain}

\newtheorem{proposition}[theorem]{Proposition}



\theoremstyle{definition}



\theoremstyle{remark}
\newtheorem*{remark}{Remark}

\newcommand{\s}[1]{\mathcal{#1}}
\newcommand{\N}{\mathbb{N}}
\newcommand{\Z}{\mathbb{Z}}
\newcommand{\Q}{\mathbb{Q}}
\newcommand{\R}{\mathbb{R}}
\newcommand{\C}{\mathbb{C}}


\mode<presentation> {
    \usetheme{Frankfurt} %Bergen, Berkely, Berlin, Boadilla, CambridgeUS, Darmstadt,
                          %Frankfurt, Goettingen, Singapore, Warsaw
    \usecolortheme{default} %beetle, seahorse, wolverine, dolphin, beaver
}

\title{Binomial Option Pricing Models}

\author{}
\date{}
\usepackage{lipsum}

\newcommand\Wider[2][3em]{%
\makebox[\linewidth][c]{%
  \begin{minipage}{\dimexpr\textwidth+#1\relax}
  \raggedright#2
  \end{minipage}%
  }%
}
% If you want to display your table of contents at every Section/Subsection:
%\AtBeginSection[] {
%\begin{frame}<beamer>{Table of Contents}
%\tableofcontents[currentsection,currentsubsection]
%\end{frame}
%}

\begin{document}

\maketitle

\begin{frame}
\frametitle{Binomial Tree}

What if we make a simple assumption about the future price of a stock:
\begin{itemize}
\item
 it can only possibly be one of two points at some point in the future, one that is  greater than the current value and one that is less than the current value. 
 \end{itemize}
 \vspace{1cm}

 Suppose the stock at time 0 is 100. Then at time 1 it can only be either 90 or 110.

\end{frame}

\begin{frame}
\frametitle{Binomial Tree}

This is drawn in a tree diagram going from left to right

  \begin{tikzpicture}[>=stealth,sloped]
  \centering
    \matrix (tree) [%
      matrix of nodes,
      minimum size=1 cm,
      column sep=2.5 cm,
      row sep=1 cm,ampersand replacement=\&
    ]
    {
          \& 110    \\
      100 \&   \\
          \& 90   \\
    };
    \draw[->] (tree-2-1) -- (tree-1-2) node [midway,above]{} ;
    \draw[->] (tree-2-1) -- (tree-3-2) node [midway,below]{};

  \end{tikzpicture}

\end{frame}

\begin{frame}
\frametitle{Binomial Option Pricing}


We will use the following notation:
\begin{table}
\centering
\begin{tabular}{lc}
\hline
$S$ & initial value of the stock \\ \hline
$u$  & \shortstack{Proportional increase of the stock \\ if it increases} \\ \hline
$d$ & \shortstack{Proportional decrease of the stock \\ if it decreases} \\ \hline
$Su$ & \shortstack{Value  of the stock at $t$ \\ if it increases} \\ \hline
$Sd$ & \shortstack{Value  of the stock at $t$ \\ if it decreases} \\ \hline
$C_u$ or $P_u$ & \shortstack{Payoff  of the option at $t$ \\ if the stock increases} \\ \hline
$C_d$ or $P_d$ & \shortstack{Payoff  of the option at $t$ \\ if the stock decreases} \\ \hline
$p$ & Probability of a movement up \\ \hline
$1-p$ & Probability of a movement down \\ \hline
\end{tabular}
\end{table}


\end{frame}

\begin{frame}
\frametitle{Binomial Option Pricing}
For example, suppose $S = 100$ and it can increase to 110 or decrease to 90 in one year and we are pricing a call option with a strike price of 105. 
\begin{itemize}
\item $Su = 110$ and $u = 1.10$,
\item $Sd = 90$ and $d = 0.90$
\item The payoff when the stock increases is $C_u = 5$,
\item The payoff when the stock decreases is $C_d = 0$. 
\end{itemize}


\end{frame}

\begin{frame}
\frametitle{Practice}

Suppose a stock worth 40 can increase to 45 or decrease to 38 in 4 months. A certain call option has a strike price of 41. Determine the values of $u$, $d$, $C_u$, and $C_d$. 

\pause

\begin{itemize}
\item $u = 45/40 = 1.125$
\item $d = 38/40 = 0.95$
\item $C_u = 4$
\item $C_d = 0$
\end{itemize}

\end{frame}


\begin{frame}
\frametitle{Risk Neutral Pricing}

Let $p$ be the probability that the stock increases to $Su$, where $0 < p < 1$. Then $1-p$ is the probability that the stock decreases to $Sd$. Then how much is a call option worth?

\begin{itemize}
\item With probability $p$ the call option is worth $C_u$.
\item With probability $1-p$ the call option is worth $C_d$.
\end{itemize}

The actuarially fair payoff of the call option is then the expected value of the payoff, $p C_u + (1-p) C_d$. The price of the call option is the present value of this \[C = e^{-rt}[p C_u + (1-p) C_d]\]

\end{frame}

\begin{frame}
\frametitle{Risk Neutral Pricing}

This method of pricing options is called risk neutral pricing and turns out to be very valuable. If $p$ is not given it can be calculated using

\[p = \frac{e^{rt}- d}{u-d}\]

There is some theory behind this that this is the only $p$ where you cannot mathematically make money with no risk (arbitrage). 

\end{frame}

\begin{frame}
\frametitle{Risk Neutral Pricing}

Going back to our example, we had a stock worth 40 with $u = 1.125$ and $d = 0.95$, $r = 0.03$ and $\delta = 0.06$. A 41 strike call had possible payoffs of $C_u = 4$ and $C_d = 0$. 

\[p =  \frac{e^{rt} - d}{u-d} = \frac{e^{(.03)(1/3)} - .95}{1.125-.95} = 0.343\]

The the call price is 

\[C = e^{-rt}[p C_u + (1-p) C_d] = e^{-.03(1/3)}[0.343(4) + 0.657 (0)] = 1.36 \]

This is the same (with some rounding error) as the previous approach.

\end{frame}

\begin{frame}
\frametitle{Risk Neutral Pricing}

Again, the same formula hold for puts, but using possible payouts for puts:

\[P = e^{-rt}[p P_u + (1-p) P_d]\]

\end{frame}

\begin{frame}
\frametitle{Practice}

A stock currently costs \$75. In 6 months it can grow to \$80.50 or fall to \$72. The interest rate is $r = 0.05$. A put option has a strike price of $K = 76.80$. Use risk neutral pricing to determine the price of the option.

\pause
\begin{itemize}
\item First we calculate all relevant values, $u = 1.073$, $d = 0.96$, $P_u = 0$ and $P_d = 4.80$. 
\pause
\item Now we find the risk neutral probability 
\[p = \frac{e^{rt} - d}{u-d} = \frac{e^{(.05)(1/2)} - .96}{1.073-.96} = 0.578\]
\pause
\item Now we calculate the put premium:
\[P = e^{-rt}[p P_u + (1-p) P_d] = e^{-.05(1/2)}[.486(0) + .514(4.80)] = 1.98\]
\end{itemize}

\end{frame}

\begin{frame}
\frametitle{Binomial Tree}

Returning to binomial trees. The following is written in terms of stock prices. 

  \begin{tikzpicture}[>=stealth,sloped]
  \centering
    \matrix (tree) [%
      matrix of nodes,
      minimum size=1 cm,
      column sep=2.5 cm,
      row sep=.5 cm,ampersand replacement=\&
    ]
    {
          \& $Su$    \\
      $S$ \&   \\
          \& $Sd$   \\
    };
    \draw[->] (tree-2-1) -- (tree-1-2) node [midway,above]{} ;
    \draw[->] (tree-2-1) -- (tree-3-2) node [midway,below]{};

  \end{tikzpicture}

\end{frame}

\begin{frame}
\frametitle{Binomial Tree}

But trees can also be written in terms of the value of the call option.

  \begin{tikzpicture}[>=stealth,sloped]
  \centering
    \matrix (tree) [%
      matrix of nodes,
      minimum size=1 cm,
      column sep=2.5 cm,
      row sep=.5 cm,ampersand replacement=\&
    ]
    {
          \& $C_u$    \\
      $C$ \&   \\
          \& $C_d$   \\
    };
    \draw[->] (tree-2-1) -- (tree-1-2) node [midway,above]{} ;
    \draw[->] (tree-2-1) -- (tree-3-2) node [midway,below]{};

  \end{tikzpicture}

$C$ is the call premium and $C_u$ and $C_d$ are the possible call payoffs. 

\end{frame}

\begin{frame}
\frametitle{Binomial Tree}

In the example where the stock price is 40 and it could increase to 45 or decrease to 38, the 41 strike option has an option tree of 

  \begin{tikzpicture}[>=stealth,sloped]
  \centering
    \matrix (tree) [%
      matrix of nodes,
      minimum size=1 cm,
      column sep=2.5 cm,
      row sep=.5 cm,ampersand replacement=\&
    ]
    {
          \& $4$    \\
      $0.90$ \&   \\
          \& $0$   \\
    };
    \draw[->] (tree-2-1) -- (tree-1-2) node [midway,above]{} ;
    \draw[->] (tree-2-1) -- (tree-3-2) node [midway,below]{};

  \end{tikzpicture}

\end{frame}

\begin{frame}
\frametitle{Two Period Binomial Tree}
Consider a \textbf{two period binomial tree}. 
\begin{tikzpicture}[>=stealth,sloped]
    \matrix (tree) [%
      matrix of nodes,
      minimum size=.3cm,
      column sep=2.5cm,
      row sep=.3cm,ampersand replacement=\&
    ]
    {
          \&   \& $Su^2$ \\
          \& $Su$ \&   \\
      \$$S$ \&   \& $Sud$ \\
          \& $Sd$ \&   \\
          \&   \& $Sd^2$ \\
    };
    \draw[->] (tree-3-1) -- (tree-2-2) node [midway,above] {};
    \draw[->] (tree-3-1) -- (tree-4-2) node [midway,below] {};
    \draw[->] (tree-2-2) -- (tree-1-3) node [midway,above] {};
    \draw[->] (tree-2-2) -- (tree-3-3) node [midway,below] {};
    \draw[->] (tree-4-2) -- (tree-3-3) node [midway,above] {};
    \draw[->] (tree-4-2) -- (tree-5-3) node [midway,below] {};
  \end{tikzpicture}
\end{frame}

\begin{frame}
\frametitle{Two Period Binomial Tree}

The idea here is that there is a step size, $h$, which in a two period binomial tree is always $h = t/2$. Then in one step size, the new stock value can be either $Su$ or $Sd$.
\begin{itemize}
\item If the stock at time $h$ is $Su$, then at time $2h = t$, the stock price can move up to $Su^2$ or down to $Sud$
\item If the stock at time $h$ is $Sd$, then at time $2h = t$, the stock price can move up to $Sdu$ or down to $Sd^2$
\end{itemize}

Because of the way we set this up, $Sud = Sdu$, so the tree is a \textbf{recombining} tree because at time $h$ the nodes combine when the paths meet up.

\end{frame}


\begin{frame}
\frametitle{Two Period Binomial Tree}

We will start with the easy way of pricing a two period binomial tree first. Recall the formula for $p$ was $p = \frac{e^{rt} - d}{u-d}$. For a multi-period tree, the risk neutral probability is \[p = \frac{e^{rh} - d}{u-d}\] The only difference is that the the $t$ is replaced with an $h$. 

\end{frame}

\begin{frame}
\frametitle{Two Period Binomial Tree}
Using this $p$ we can find the probability of ending up in certain places. 

\begin{tikzpicture}[>=stealth,sloped]
    \matrix (tree) [%
      matrix of nodes,
      minimum size=.3cm,
      column sep=2.5cm,
      row sep=.5cm,ampersand replacement=\&
    ]
    {
          \&   \& $Su^2$ \\
          \& $Su$ \&   \\
      \$$S$ \&   \& $Sud$ \\
          \& $Sd$ \&   \\
          \&   \& $Sd^2$ \\
    };
    \draw[->] (tree-3-1) -- (tree-2-2) node [midway,above] {$p$};
    \draw[->] (tree-3-1) -- (tree-4-2) node [midway,below] {$1-p$};
    \draw[->] (tree-2-2) -- (tree-1-3) node [midway,above] {$p^2$};
    \draw[->] (tree-2-2) -- (tree-3-3) node [midway,below] {$(1-p)p$};
    \draw[->] (tree-4-2) -- (tree-3-3) node [midway,above] {};
    \draw[->] (tree-4-2) -- (tree-5-3) node [midway,below] {$(1-p)^2$};
  \end{tikzpicture}
\end{frame}

\begin{frame}
\frametitle{Two Period Binomial Tree}

We know that 
\begin{itemize}
\item There is a $p^2$ probability of ending up at $Su^2$
\item a $2p(1-p)$ probability of ending up at $Sud$ 
\item a $(1-p)^2$ probability of ending up at $Sd^2$
\end{itemize}

\end{frame}

\begin{frame}
\frametitle{Two Period Binomial Tree}
Now consider the option-based binomial tree with two periods
\begin{tikzpicture}[>=stealth,sloped]
    \matrix (tree) [%
      matrix of nodes,
      minimum size=.3cm,
      column sep=2.5cm,
      row sep=.5cm,ampersand replacement=\&
    ]
    {
          \&   \& $C_{u^2}$ \\
          \& $C_u$ \&   \\
      \$$C$ \&   \& $C_{ud}$ \\
          \& $C_d$ \&   \\
          \&   \& $C_{d^2}$ \\
    };
    \draw[->] (tree-3-1) -- (tree-2-2) node [midway,above] {$p$};
    \draw[->] (tree-3-1) -- (tree-4-2) node [midway,below] {$1-p$};
    \draw[->] (tree-2-2) -- (tree-1-3) node [midway,above] {$p^2$};
    \draw[->] (tree-2-2) -- (tree-3-3) node [midway,below] {$(1-p)p$};
    \draw[->] (tree-4-2) -- (tree-3-3) node [midway,above] {};
    \draw[->] (tree-4-2) -- (tree-5-3) node [midway,below] {$(1-p)^2$};
  \end{tikzpicture}
\end{frame}

\begin{frame}
\frametitle{Two Period Binomial Tree}

We know that 
\begin{itemize}
\item $C_{u^2}$ is the payoff of the option if the stock price at $2h$ is $Su^2$ ($max(0,Su^2 - K)$)
\item $C_{ud}$ is the payoff of the option if the stock price at $2h$ is $Sud$ ($max(0,Sud - K)$)
\item $C_{d^2}$ is the payoff of the option if the stock price at $2h$ is $Sd^2$ ($max(0,Sd^2 - K)$)
\item $C_u$ is the value of the call option position at time $h$ is the stock price is $S_u$. 
\item $C_d$ is the value of the call option position at time $h$ is the stock price is $S_d$. 
\item $C$ is the option premium
\end{itemize}

\end{frame}

\begin{frame}
\frametitle{Two Period Binomial Tree}

 $C_{u^2}$, $C_{ud}$ and $C_{d^2}$ are easily determine using the payoff of the call options at the particular stock prices. For example, suppose a stock costs \$100 amd $u = 1.04$ and $d = 0.97$. The stock tree would be 

\begin{tikzpicture}[>=stealth,sloped]
    \matrix (tree) [%
      matrix of nodes,
      minimum size=.3cm,
      column sep=2.5cm,
      row sep=.3cm,ampersand replacement=\&
    ]
    {
          \&   \& $108.16$ \\
          \& $104$ \&   \\
      $100$ \&   \& $100.88$ \\
          \& $97$ \&   \\
          \&   \& $94.09$ \\
    };
    \draw[->] (tree-3-1) -- (tree-2-2) node [midway,above] {};
    \draw[->] (tree-3-1) -- (tree-4-2) node [midway,below] {};
    \draw[->] (tree-2-2) -- (tree-1-3) node [midway,above] {};
    \draw[->] (tree-2-2) -- (tree-3-3) node [midway,below] {};
    \draw[->] (tree-4-2) -- (tree-3-3) node [midway,above] {};
    \draw[->] (tree-4-2) -- (tree-5-3) node [midway,below] {};
  \end{tikzpicture}

\end{frame}

\begin{frame}
\frametitle{Two Period Binomial Tree}
Suppose a certain call option had a strike price of 98. 
 \begin{itemize}
 \item If the stock price is 108.16, then the payoff is $108.16 - 98 = 10.16$
 \item If the stock price is 100.88, then the payoff is $100.88 - 98 = 2.88$
 \item If the stock price is 94.09, then the payoff is 0, because the option is not exercised.
 \end{itemize}
\end{frame}

\begin{frame}
\frametitle{Two Period Binomial Tree}

 \begin{itemize}
 \item That means there is a $p^2$ probability of a payoff of 10.16,
 \item a $2p(1-p)$ probability of a payoff of 2.88
 \item and a $(1-p)^2$ probability of a payoff of 0. 
 \end{itemize}

The risk neutral price is then $C = e^{-rt}[p^2 (10.18) + 2p(1-p) (2.88) + (1-p)^2 (0)]$

\end{frame}

\begin{frame}
\frametitle{Two Period Binomial Tree}

In general, the premium for a call option using a two period binomial tree is \[C = e^{-rt}[p^2 C_{u^2} + 2p(1-p) C_{ud} + (1-p)^2 C_{d^2}]\]

The premium for a put option using a two period binomial tree is \[P = e^{-rt}[p^2 P_{u^2} + 2p(1-p) P_{ud} + (1-p)^2 P_{d^2}]\]

\end{frame}

\begin{frame}
\frametitle{Example}

The put option payoffs are $P_{u^2} = 0$, $P_{ud} = 0.417$, and $P_{d^2} = 2.953$. 

\vspace{.3 cm} 

We must now find $p$. \[p = \frac{e^{r h} - d}{u-d} = \frac{e^{(.04) (1/12)} - .975}{1.027-0.975} = 0.545\]

\vspace{.3 cm}

The put option premium is  \[P = e^{-.04 (1/6)}[(0.545)^2 0 + 2(.545)(1-.545)(0.417) + (1-0.545)^2 2.953] \]
\[ = 0.81\]

\end{frame}

\begin{frame}
\frametitle{Multi-Period Binomial Tree}

\begin{tikzpicture}[>=stealth,sloped]
    \matrix (tree) [%
      matrix of nodes,
      minimum size=.3cm,
      column sep=1.5cm,
      row sep=.3cm,ampersand replacement=\&
    ]
    {
    	\&  \&  \&  $Su^3$ \\
          \&   \& $Su^2$ \& \\
          \& $Su$ \&  \& $Su^2d$ \\
      $S$ \&   \& $Sud$ \& \\
          \& $Sd$ \&  \& $Sud^2$ \\
          \&   \& $Sd^2$  \& \\
          \&  \&  \&  $Sd^3$ \\
    };
    \draw[->] (tree-4-1) -- (tree-3-2) node [midway,above] {};
    \draw[->] (tree-4-1) -- (tree-5-2) node [midway,below] {};
    \draw[->] (tree-3-2) -- (tree-2-3) node [midway,above] {};
    \draw[->] (tree-3-2) -- (tree-4-3) node [midway,below] {};
    \draw[->] (tree-5-2) -- (tree-4-3) node [midway,above] {};
    \draw[->] (tree-5-2) -- (tree-6-3) node [midway,below] {};
    \draw[->] (tree-4-3) -- (tree-3-4) node [midway,above] {};
    \draw[->] (tree-4-3) -- (tree-5-4) node [midway,below] {};
    \draw[->] (tree-2-3) -- (tree-1-4) node [midway,above] {};
    \draw[->] (tree-6-3) -- (tree-7-4) node [midway,below] {};
    \draw[->] (tree-6-3) -- (tree-5-4) node [midway,below] {};
    \draw[->] (tree-2-3) -- (tree-3-4) node [midway,above] {};

  \end{tikzpicture}


\end{frame}

\begin{frame}
\frametitle{Multi-Period Binomial Tree}

\begin{tikzpicture}[>=stealth,sloped]
    \matrix (tree) [%
      matrix of nodes,
      minimum size=.3cm,
      column sep=1.5cm,
      row sep=.3cm,ampersand replacement=\&
    ]
    {
    	\& \& \& \& $Su^4$ \\
    	\&  \&  \&  $Su^3$ \& \\
          \&   \& $Su^2$ \&  \& $Su^3d$ \\
          \& $Su$ \&  \& $Su^2d$  \& \\
      $S$ \&   \& $Sud$ \&  \& $Su^2d^2$\\
          \& $Sd$ \&  \& $Sud^2$ \& \\
          \&   \& $Sd^2$  \&  \& $Sud^3$ \\
          \&  \&  \&  $Sd^3$  \& \\
          \& \& \& \& $Sd^4$ \\
    };
    \draw[->] (tree-5-1) -- (tree-4-2) node [midway,above] {};
    \draw[->] (tree-5-1) -- (tree-6-2) node [midway,below] {};
    \draw[->] (tree-4-2) -- (tree-3-3) node [midway,above] {};
    \draw[->] (tree-4-2) -- (tree-5-3) node [midway,below] {};
    \draw[->] (tree-6-2) -- (tree-5-3) node [midway,above] {};
    \draw[->] (tree-6-2) -- (tree-7-3) node [midway,below] {};
    \draw[->] (tree-5-3) -- (tree-4-4) node [midway,above] {};
    \draw[->] (tree-5-3) -- (tree-6-4) node [midway,below] {};
    \draw[->] (tree-3-3) -- (tree-2-4) node [midway,above] {};
    \draw[->] (tree-7-3) -- (tree-8-4) node [midway,below] {};
    \draw[->] (tree-2-4) -- (tree-1-5) node [midway,above] {};
    \draw[->] (tree-2-4) -- (tree-3-5) node [midway,below] {};
    \draw[->] (tree-4-4) -- (tree-3-5) node [midway,above] {};
    \draw[->] (tree-4-4) -- (tree-5-5) node [midway,below] {};
    \draw[->] (tree-6-4) -- (tree-5-5) node [midway,above] {};
    \draw[->] (tree-6-4) -- (tree-7-5) node [midway,below] {};
    \draw[->] (tree-8-4) -- (tree-7-5) node [midway,above] {};
    \draw[->] (tree-8-4) -- (tree-9-5) node [midway,below] {};
    \draw[->] (tree-7-3) -- (tree-6-4) node [midway,below] {};
    \draw[->] (tree-3-3) -- (tree-4-4) node [midway,above] {};
  \end{tikzpicture}


\end{frame}


\begin{frame}
\frametitle{Multi-Period Binomial Tree}

The probability of getting to a stock price $Su^jd^k$, which is the probability of $j$ up movements and $k$ down movements, is \[Pr(S_t = Su^jd^k) = \left(\begin{array}{c} j+k \\ k \end{array}\right) p^j(1-p)^k\]


\end{frame}

\begin{frame}
\frametitle{Multi-Period Binomial Tree}

This means that the price of a call option using a $m$-period binomial tree is  \[C = e^{-rt}\sum_{k=0}^m \left(\begin{array}{c} m \\ k \end{array}\right) p^{m-k}(1-p)^k C_{u^{m-k}d^k}\]


\end{frame}


\begin{frame}
\frametitle{Example}

Suppose a stock costs \$50 and the binomial tree has $h = .25$, $t = 1$, $u = 1.05$ and $d = 0.95$,  and $p = 0.60$. What is the price of the 1 year at the money call option. Assume $r=0$. 

\begin{itemize}
\item $Su^4 = 60.775$ so $C_{u^4} = 10.775$
\item $Su^3 d = 54.987$ so $C_{u^3d} = 4.987$
\item $Su^2d^2 = 49.75$ so $C_{u^2d^2} = 0$ as do all the other payoffs, ($C_ud^3$ and $C_d^4$)
\end{itemize}

So the call premium is \[C = (.6^4)(10.775) + 4(.6^3)(.4)4.987 + 0 = 3.12\]


\end{frame}

\begin{frame}
\frametitle{Practice}

Determine the 6 month 145-strike call option premium using a 3 period binomial tree with $S = 140$, $r = 0.06$, $u = 1.0659$, $d = 0.9508$, and $ p = 0.4857$. 

\pause 
The binomial tree for the stock price is 


\begin{tikzpicture}[>=stealth,sloped]
    \matrix (tree) [%
      matrix of nodes,
      minimum size=.3cm,
      column sep=1.5cm,
      row sep=.2cm,ampersand replacement=\&
    ]
    {
    	\&  \&  \&  $169.54$ \\
          \&   \& $159.06$ \& \\
          \& $149.23$ \&  \& $151.23$ \\
      $140$ \&   \& $141.88$ \& \\
          \& $133.11$ \&  \& $134.89$ \\
          \&   \& $126.55$  \& \\
          \&  \&  \&  $120.32$ \\
    };
    \draw[->] (tree-4-1) -- (tree-3-2) node [midway,above] {};
    \draw[->] (tree-4-1) -- (tree-5-2) node [midway,below] {};
    \draw[->] (tree-3-2) -- (tree-2-3) node [midway,above] {};
    \draw[->] (tree-3-2) -- (tree-4-3) node [midway,below] {};
    \draw[->] (tree-5-2) -- (tree-4-3) node [midway,above] {};
    \draw[->] (tree-5-2) -- (tree-6-3) node [midway,below] {};
    \draw[->] (tree-4-3) -- (tree-3-4) node [midway,above] {};
    \draw[->] (tree-4-3) -- (tree-5-4) node [midway,below] {};
    \draw[->] (tree-2-3) -- (tree-1-4) node [midway,above] {};
    \draw[->] (tree-6-3) -- (tree-7-4) node [midway,below] {};
        \draw[->] (tree-6-3) -- (tree-5-4) node [midway,below] {};
    \draw[->] (tree-2-3) -- (tree-3-4) node [midway,above] {};
  \end{tikzpicture}


\end{frame}

\begin{frame}
\frametitle{Practice}


\vspace{.1 cm}
$C_{u^3} = 24.54$, $C_{u^2d} = 6.23$, $C_{ud^2} = 0$.

\vspace{.2 cm}
The call premium is \[C = e^{-.06/2}[(0.4857)^3(24.54) + 3(.4857)^2(.5143)(6.23)] = 4.93\]


\end{frame}

\begin{frame}
\frametitle{Practice}

Determine the price of a 46 strike 4 month put option using a 4 period binomial tree with $S = 50$, $r = 0.03$, $u = 1.022$, $d = 0.976$, and $p = 0.494$. 

\vspace{.1 cm}
\pause  $Sd^4 = 45.285$ and $Sud^3 = 47.43$, and all the rest are higher than 46, so the payoff is 0 except for $Sd^4$. 

\vspace{.1 cm}

\[P = e^{-.03/3}[(1-.494)^4(.715)] = .046\]

\end{frame}



\end{document}







