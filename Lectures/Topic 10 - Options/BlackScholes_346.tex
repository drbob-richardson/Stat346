\documentclass{beamer}
\usepackage{beamerinnerthemecircles, beamerouterthemeshadow}
\usepackage{amsmath, amsfonts, amscd, epsfig, amssymb, amsthm}
\usepackage{amsmath, amssymb, amsfonts, amsthm, epsfig, graphicx,float,enumerate}
\usepackage{graphicx,float,enumerate,natbib, subfigure, caption,dsfont}
\usepackage{graphicx,float,enumerate,relsize,bm,subfigure}
\usepackage[gen]{eurosym}
\usepackage{siunitx}

\DeclareSIUnit{\pers}{pers}
\DeclareSIUnit{\EUR}{\text{\euro}}
\sisetup{
  per-mode = fraction,
  inter-unit-product = \ensuremath{{}\cdot{}},
}

\usepackage{tikz}
\usetikzlibrary{matrix}




\mode<presentation> {
    \usetheme{Frankfurt} %Bergen, Berkely, Berlin, Boadilla, CambridgeUS, Darmstadt,
                          %Frankfurt, Goettingen, Singapore, Warsaw
    \usecolortheme{default} %beetle, seahorse, wolverine, dolphin, beaver
}

\title{Black Scholes}

\author{}
\date{}
\usepackage{lipsum}

\newcommand\Wider[2][3em]{%
\makebox[\linewidth][c]{%
  \begin{minipage}{\dimexpr\textwidth+#1\relax}
  \raggedright#2
  \end{minipage}%
  }%
}
% If you want to display your table of contents at every Section/Subsection:
%\AtBeginSection[] {
%\begin{frame}<beamer>{Table of Contents}
%\tableofcontents[currentsection,currentsubsection]
%\end{frame}
%}

\begin{document}

\maketitle

\begin{frame}
\frametitle{Binomial Tree Limit}

Consider a 100-strike option with $S=100$, $r = .06, \delta = .06, \sigma = .1$ and $t=1$. What happens as the number of binomial periods increases?

\begin{table}
\centering
\begin{tabular}{|c|c|}
\hline
Num of Periods & Premium \\ \hline
1 & 4.70 \\ \hline
2 & 3.32 \\ \hline
3 & 4.07 \\ \hline
4 & 3.53 \\ \hline
5 & 3.94 \\ \hline
6 & 3.60 \\ \hline
\end{tabular}
\end{table}

\end{frame}

\begin{frame}
\frametitle{Binomial Tree Limit}

Continued ...
\begin{table}
\centering
\begin{tabular}{|c|c|}
\hline
Num of Periods & Premium \\ \hline
7 & 3.89 \\ \hline
8 & 3.64 \\ \hline
9 & 3.86 \\ \hline
10 & 3.66 \\ \hline
11 & 3.84 \\ \hline
12 & 3.67 \\ \hline
$\vdots$ & $\vdots$ \\ \hline
50 & 3.73 \\ \hline
100 & 3.75 \\ \hline
1000 & 3.755 \\ \hline
\end{tabular}
\end{table}

\end{frame}

\begin{frame}
\frametitle{Binomial Tree Limit}

Eventually it settles near a specific value. As fun as it would be to
do a 1000 period binomial tree, there is an easier way to  find out
what that limit is.

\vspace{2 cm}

A set of equations gives us the limit. They are called the \textbf{Black-Scholes} equations.

\end{frame}

\begin{frame}
\frametitle{Black-Scholes}
The Black-Scholes equations:
\begin{eqnarray*}
d_1 &=& \frac{\log(S/K) + (r+.5 \sigma^2) T}{\sigma \sqrt{T}} \\
d_2 &=& d_1-\sigma\sqrt{T} \\
C &=& S N(d_1) - K e^{-rT} N(d_2) \\
P &=& K e^{-rT} N(-d_2) - S N(-d_1) \\
\end{eqnarray*}

where $N(\cdot)$ is the normal CDF function, found from a table.
\end{frame}

\begin{frame}
\frametitle{Black-Scholes}
Consider a 100-strike option with \(S=100\), \(r = .06\), \(\sigma = .1\) and \(T=1\). Assume dividend rate \(\delta = 0\). What is the Black-Scholes price?
\pause
\begin{eqnarray*}
d_1 &=& \frac{\log(S/K) + (r + .5 \sigma^2) T}{\sigma \sqrt{T}} \\
    &=& \frac{\log(100/100) + (.06 + .5 \times .1^2) \times 1}{.1 \times \sqrt{1}} \\
    &=& 0.65 \\
d_2 &=& d_1 - \sigma\sqrt{T} \\
    &=& 0.65 - 0.1 \times \sqrt{1} \\
    &=& 0.55 \\
\end{eqnarray*}

\end{frame}

\begin{frame}
\frametitle{Black-Scholes}
Consider a 100-strike option with \(S=100\), \(r = .06\), \(\sigma = .1\) and \(T=1\). What is the Black-Scholes price?
\begin{eqnarray*}
d_1 &=&  0.65 \\
d_2 &=&  0.55 \\
\end{eqnarray*}
The call option price is:
\begin{eqnarray*}
C &=& 100 N(d_1) - 100 e^{-0.06} N(d_2) \\
  &=& 100 N(0.65) - 100 e^{-0.06} N(0.55) \\
  &=& 7.459322
\end{eqnarray*}
\end{frame}


\begin{frame}
\frametitle{Black-Scholes Practice}
The current price of a stock is \$40,  the risk free rate is \(r = .03\), and the volatility of the stock is \(\sigma = .1\). Using Black-Scholes, what is the price of a call option that expires in 9 months to purchase the stock at a strike price of 39?
\begin{eqnarray*}
d_1 &=& 0.595 \\
d_2 &=& 0.509\\
\end{eqnarray*}
\pause
The call option price is:
\begin{eqnarray*}
C &=& 40 N(d_1) - 39 e^{-0.03 \cdot 0.75} N(d_2) \\
&=& 2.483579
\end{eqnarray*}
\end{frame}

\begin{frame}
\frametitle{Black-Scholes Practice for Put Option}
Using the same parameters, what is the price of a \emph{put} option?
\pause
The put option price is:
\begin{eqnarray*}
P &=& 39 e^{-0.03 \cdot 0.75} N(-d_2) - 40 N(-d_1) \\
&=& 0.6158774
\end{eqnarray*}
\end{frame}


\end{document}







