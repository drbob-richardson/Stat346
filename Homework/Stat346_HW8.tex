\documentclass{article}
\usepackage{setspace,fullpage,xcolor}
\usepackage{booktabs} % For formal tables
\usepackage{tabularx} % For adjustable-width columns
\begin{document}

\begin{center}
\textbf{Stat 346 Homework \#8}\\
\vspace{0.3in}
\end{center}

\section*{Homework Questions}

\begin{enumerate}
   \item Given the partial paid claims triangle below, fill in the missing parts using the chain ladder method. Assume claims develop over a period of four years and that development factors are consistent over time. To find loss development factors, use an arithmetic mean. 

\[
\begin{tabular}{c|cccc}
Accident Year & Year 1 & Year 2 & Year 3 & Year 4 \\
\hline
2016 & 10,000 & 5,000 & 3,000 & 1,000 \\
2017 & 12,000 & 5,500 & 2,000 & - \\
2018 & 13,000 & 6,000 & - & - \\
2019 & 14,000 & - & - & - \\
\end{tabular}
\]
   
       \item For the cumulative claims triangle below, calulate the loss development factor for going from development year 1 to development year 2 using three different methods: volume weighted mean \textcolor{red}{[1.535]}, xHiLow mean  \textcolor{red}{[1.580]}, and geometric mean  \textcolor{red}{[1.520]}.
    
    \[
    \begin{tabular}{c|ccccc}
    Accident Year & Year 1 & Year 2 & Year 3 & Year 4 & Year 5\\
    \hline
    2016 & 9,000 & 15,000 & 18,000 & 21,000  & 22,000 \\
    2017 & 12,000 & 19,000 & 22,500 & 24,000 & - \\
    2018 & 13,000 & 20,500 & 24,500 & - & - \\
    2019 & 15,000 & 20,000 & - & - & - \\
    2020 & 15,000 & - & - & - & - \\
    \end{tabular}
    \]
    
    \item The loss development factors are shown for four years in the following table. 
    
    \[
    \begin{tabular}{c|c}
    Development Year Ratio & LDF \\
    \hline
    1/0 & 1.5 \\
    2/1 &1.3\\
    3/2 & 1.15\\
    4/3 & 1.1 \\
    $\infty$/4 & 1.02
    \end{tabular}
    \]
    We know the following
    Given the following data:
\begin{itemize}
    \item Expected loss ratio: 60\%
    \item Earned premiums for the current year: \$1,500,000
    \item Total paid for losses in the current calendar year: \$410,000
\end{itemize}

    Using the chain ladder method \textcolor{red}{[621,594]}, expected claims method \textcolor{red}{[490,000]}, and Bornhuetter-Ferguson method \textcolor{red}{[542,301]}, calculate the reserves needed for the current year.

\newpage

    \item
    
    
\vspace{0.3in}

This table outlines payments made and case reserves set aside for accident years across different calendar years.

\[
\begin{tabular}{cccccc}
\toprule
\textbf{Calendar Year} & \textbf{Accident Year} & \textbf{Payments Made} & \textbf{Case Reserves} \\
\midrule
2015 & 2015 & \$50,000 & \$20,000 \\
\midrule
2016 & 2015 & \$10,000 & \$15,000 \\
2016 & 2016 & \$60,000 & \$36,000 \\
\midrule
2017 & 2015 & \$6,000 & \$10,000 \\
2017 & 2016 & \$30,000 & \$14,000 \\
2017 & 2017 & \$55,000 & \$32,000 \\
\midrule
2018 & 2015 & \$5,000 & \$5,000 \\
2018 & 2016 & \$10,000 & \$10,000 \\
2018 & 2017 & \$20,000 & \$18,000 \\
2018 & 2018 & \$70,000 & \$25,000 \\
\bottomrule
\end{tabular}
\]

Create 2 different claims triangle based on this information. One for reported claims, one for paid claims. 


\item Discuss the differences between a claims triangle that uses paid claims data versus one that uses reported claims data. What is the difference between what you would call the unknown parts of the claims triangle. 

\end{enumerate}

\end{document}
