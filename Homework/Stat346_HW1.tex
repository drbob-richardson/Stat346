\documentclass[11pt]{article}
\usepackage{amsmath,amssymb,amsfonts,amsthm,epsfig,bm}
\usepackage{graphicx,float,enumerate,natbib}
\usepackage[titletoc,title]{appendix}
%\usepackage[nomarkers,nolists]{endfloat}
\usepackage[]{color}
\usepackage{bm}

\newtheorem{result}{Result}
\newtheorem{lemma}{Lemma}
%
%\renewcommand{\topfraction}{0.95}
%\renewcommand{\textfraction}{0.05}
%\renewcommand{\floatpagefraction}{0.95}
%\renewcommand{\floatpagefraction}{0.9}
%
\usepackage{setspace}
\setlength{\topmargin}{-0.3in}
\setlength{\oddsidemargin}{0.3in}
\setlength{\textwidth}{6.31in}
\setlength{\textheight}{8.5in}
%
\newcommand{\bs}{\bf{blue}}
%
%


\pdfminorversion=4

\begin{document}                                                                                                                                                                                                                                                                                                                                                                                                                                                                                                                                                                                                                                                     
\title{Stat 346: Homework 1}

\date{}

\maketitle

    \begin{enumerate}
        \item Let $X$ be continuous with pdf:
        \[f(x) = 3x^2, 0 < x < 1.\]
        Find:
        \begin{enumerate}
            \item $E(X)$ 
            \item $Var(X)$
            \item $E(X^k)$
            \item $E(3X - 5X^2 + 1)\\\\
            Answers: a=\frac{3}{4}\,\, b=\frac{3}{80} \,\, c=\frac{3}{k+3} \,\, d=\frac{1}{4} $
        \end{enumerate}
        \item Show that the following relationship holds:
        \[E(X)=e(d)S(d) + E(X \wedge d).\]
        %\item Using the moments, limiting tail behavior, and hazard functions compare the tail weight of the Weibull and inverse Weibull distributions.
        \item Assume random variable $X$ has the following pdf,
        \[f(x) = (1 + 2x^2)e^{-2x}, x \geq 0.\]
        Determine the following:
            \begin{enumerate}
                \item $S(x)$
                %\item $h(x)$
                \item $e(x)$
                \item $\lim_{x\rightarrow \infty}e(x) \\\\ %$\lim_{x\rightarrow \infty}h(x)$ and 
                Answers: a=e^{-2x}(x^2+x+1) 
                %\,\, b=\frac{2x^2 + 1}{x^2 + x + 1} 
                \,\, b=\frac{\frac{1}{2}x^2 + x + 1}{x^2 + x + 1} \,\, c= \frac{1}{2} $
            \end{enumerate}
	\item For a random variable $X$ you are given that 
	\begin{enumerate} 
	   \item[(i)] The mean is 4.
            \item[(ii)]The variance is 2. 
            \item[(iii)] The raw third moment is 3. 
        \end{enumerate}
	 Determine the coefficient of skewness of $X$. \textcolor{red}{[-30.052]}
	\item A Pareto distribution has parameters $\alpha$ = 4 and $\theta$ = 2. Determine its skewness. \textcolor{red}{[7.071]}
	\item Claim size for an insurance coverage follows a lognormal distribution with mean 1000 and median 800. Determine the probability that a claim will be greater than 1200. [0.2709]
	\item Let each $Y_i, i=1,\ldots,n$ follows a $Gam(2,4)$ distribution and are independent. 
		\begin{enumerate}
			\item Derive, using the pdf, $E(Y_i^4)$. \textcolor{red}{[30720]}
			\item Further assume that $Z=\sum_{i=1}^n Y_i$, find $Var(Z)$. \textcolor{red}{[32n]}
		\end{enumerate}



\item Using the distribution function \[F_X(x) = \begin{cases} 0 & \mbox{for} ~~ x < 0 \\ .001 x^3 & \mbox{for} ~~ 0 \leq x < 10 \\ 1 & \mbox{for}~~ x \geq 10 \end{cases}\] determine the difference between the mean excess loss function at $d=1$ and the mean of the left censored and shifted random variable with $d=1$.  

\end{enumerate}

\end{document}


