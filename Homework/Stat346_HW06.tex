\documentclass{article}
\usepackage{setspace,fullpage,xcolor}


\begin{document}

\begin{center}
\textbf{Stat 346 Homework \#7}\\
Due: Wednesday March 22nd in class.
\vspace{0.3in}
\end{center}
	\begin{enumerate}
		\item You are given the following losses:
			\begin{center}
				800, 1100, 1300, 2150, 2300, 2450, 2650
			\end{center}
			You fit an inverse exponential to the loss distribution using maximum likelihood.\\
			Determine the resulting estimate of the probability of a loss below 1000. \textcolor{red}{[.219]}
		\item The following claim experience is observed:
		\begin{center}
			\begin{tabular}{c|c}
				Claim Size & Number of Claims\\
				\hline
				0 - 1500 & 15\\
				1500 - 3000 & 8\\
				3000 - $\infty$ & 3
			\end{tabular}
		\end{center}
			You fit an exponential to the claim size distribution using maximum likelihood.\\
			Show that 1543.431 is the maximum likelihood estimate for mean claims. 
		\item You are given:
			\begin{itemize}
				\item Fourteen lives are subject to the survival function
					\begin{center}
						$S(t) = (1-\frac{t}{k})^\frac{1}{2}, \qquad 0 \leq t \leq k.$
					\end{center}
				\item The first two deaths in the sample occurred at time t = 12.
				\item The study ends at time t = 12. \\
				Find the maximum likelihood estimate of k. \textcolor{red}{[42]}
			\end{itemize}
		\item 
Consider claims following a distribution with the probability density function (PDF) given by:
\[
f(x|\theta) = 2\theta x e^{-\theta x^2}, \quad x > 0
\]
where \(\theta\) is an unknown parameter.

Given the following claim amounts:
\begin{center}
100, 150, 200, 250, 300, 350, 400
\end{center}
Calculate the Value-at-Risk (VaR) at the 95\% confidence level for a new claim based on the most. likely value of the unknown parameter. 
\item A study on the duration of a particular medical treatment's effect is conducted. The treatment effectiveness duration follows an exponential distribution with mean \(\theta\). However, the study faces two constraints:

\begin{enumerate}
    \item Only patients who experience treatment effectiveness for more than 1 month are included in the study (left truncation at 1 month).
    \item For some patients, the treatment is still effective at the end of the study, marking these data points as right-censored at 6 months.
\end{enumerate}

Given the following observed effectiveness durations (in months) and noting that two observations are right-censored:
\begin{center}
2, 3, 4, (6+), (6+)
\end{center}
Determine the MLE for the mean duration \(\theta\) of the treatment's effectiveness. 




	\end{enumerate}
\end{document}