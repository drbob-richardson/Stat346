\documentclass{article}
\usepackage{setspace,fullpage,xcolor}


\begin{document}

\begin{center}
\textbf{Stat 346 Homework \#4}\\
\vspace{0.3in}
\end{center}
\begin{enumerate}
\item
    Assume that the annual losses on each individual policy are either 0 with probability 0.2 or follow an exponential distribution with a mean of 1000 with probability 0.8. There are 2500 independent, identically distributed policyholders in a book of business.
    \begin{enumerate}
        \item What is the expected total loss from the book of business? \textcolor{red}{[2,000,000]}
        \item What is the variance of that total loss? \textcolor{red}{[2,400,000,000]}
        \item Using the normal approximation, what is the probability that the total loss is greater than 2,100,000 (the true probability is approximately 0.022)? \textcolor{red}{[.021]}
        \item Comment on how close you expect the normal approximation and the true probability to be in this example 
    \end{enumerate}
    
 \item 
  Given that $X_i \sim N(\mu_i, \sigma_i^2)$ for $i=1,2,\ldots,n$ are independent normal random variables, prove that $S = \sum_{i=1}^{n} X_i$ is also normally distributed. Use the moment generating function (MGF) of the normal distribution, which is $M_X(t) = e^{\mu t + \frac{1}{2} \sigma^2 t^2}$, in your proof.

\item
  Consider a portfolio of 1000 independent policies. Each policy has a 10\% chance of incurring a loss, which, when it occurs, follows a gamma distribution with $\alpha = 3$ and $\theta = 500$. Using a normal approximation, what is TVaR$_{.95}$ for aggregate losses? \textcolor{red}{[184,361.35]}

\item 
Let $S = \sum_{i=1}^N X_i$ be the aggregate loss, where $N$ is the number of claims following a Poisson distribution with parameter $\lambda = 2$, and $X_i$ are i.i.d. exponential random variables with mean 1000.
    \begin{enumerate}
        \item    Compute the mean and variance of $S$.
        \item Would this be a situation where you would use a normal approximation?
    \end{enumerate}


\item 
	The cumulative loss distribution for a risk $X_i$ is 
	\[F_i(x) = 1 - \frac{10^9}{(x+10^3)^3} \qquad x > 0\]
for all $i$.   Assume $S = \sum_{i=1}^N X_i$ where $N$ follows a negative binomial distribution with  distribution with parameter $r = 4$ and $\beta = 2.5$. Calculate the mean and variance of $S$. 

\item 
  Consider a risk model where $N$, the number of claims, and $X_i$, the amount of each claim, are both discrete random variables. The probability distribution of $N$ and $X$ are given in the following tables:

  \begin{table}[h]
  \centering
  \begin{minipage}{.5\linewidth}
  \centering
  \begin{tabular}{cc}
  \hline
  $n$ & ${Pr}(N=n)$ \\ \hline
  0 & 0.5 \\
  1 & 0.3 \\
  2 & 0.2 \\ \hline
  \end{tabular}
  \end{minipage}%
  \begin{minipage}{.5\linewidth}
  \centering
 
  \begin{tabular}{cc}
  \hline
  $x$ & ${Pr}(X=x)$ \\ \hline
  100 & 0.5 \\
  200 & 0.3 \\ 
  300 & 0.2 \\ \hline
  \end{tabular}
  \end{minipage}
  \end{table}

  Let $S = \sum_{i=1}^{N} X_i$ represent the total claim amount. Calculate ${Pr}(S < 500)$.



%\item 
%	The distribution for claim severity follows a single parameter Pareto distribution of the following form:
%	\[f(x) = \left( \frac{3}{1000}\right)\left( \frac{x}{1000}\right)^{-4} \qquad x > 1000\]
%	Determine the average size of a claim between 10,000 and 100,000, given that the claim is between 10,000 and 100,000. \textcolor{red}{[14,864.86]}
%\item 
%	The cumulative loss distribution for a risk is 
%	\[F(x) = 1 - \frac{10^6}{(x+10^3)^2} \qquad x > 0\]
%	An insurance policy pays the loss subject to a deductible of 1,000 and a maximum covered loss of 10,000. Calculate the percentage of the total loss amount that is expected to be paid by the company. \textcolor{red}{[40.91\%]}
%\item Losses follow a mixture distribution. There is a 30\% chance of 0 claims. Given the claims are greater than 0, they are uniformly distributed on [100,1000]. Your insurance has a deductible of 250 and a coinsurance rate of 80\%. Coinsurance applies after the deductible. Calculate the variances of the amount paid per loss and per payment.  \textcolor{red}{[39,375 , 30,000]}
%\item 
%	There are two types of policyholders, high risk and low risk. Both types have exponentially distributed claims but with different parameters ($\theta = 100$ for low risk and $\theta = 1,000$ for high risk). In the population at large, 20\% of people are high risk. 
%	
%	Calculate 
%	\begin{enumerate}
%		\item The expected value and variance for a randomly selected policyholder. \textcolor{red}{[280, 337,600]}
%		\item The expected value and variance for 3,000 randomly selected policyholders. \textcolor{red}{[840,000 , 1,012,800,000]}
%		\item The expected value and variance for a group of 3,000 policyholders, of which 600 are high risk and the other 2,400 are low risk. \textcolor{red}{[840,000 , 624,000,000]}
%		\item Explain why the answers in (b) and (c) are different.
%		\item Out of the 3,000 policyholders from (c), you randomly select 4. What is the probability that at least 3 of those 4 policyholders are high risk? \textcolor{red}{[.0271]}
%	\end{enumerate}
\end{enumerate}
\end{document} 