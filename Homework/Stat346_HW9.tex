\documentclass{article}
\usepackage{setspace,fullpage,xcolor}
\usepackage{booktabs} % For formal tables
\usepackage{tabularx} % For adjustable-width columns
\begin{document}

\begin{center}
\textbf{Stat 346 Homework \#9}\\
\vspace{0.3in}
\end{center}

\section*{Homework Questions}

\begin{enumerate}
   \item Policy A was written prior to 2010. Policy C was written on 01/01/2011, but individuals purchase 1 year policies at any time during the year. Find the following using the claim detail from the table below:
\begin{itemize}
    \item AY 2010 reported (incurred) claims @ 12/31/2010
    \item AY 2010 reported (incurred) claims @ 12/31/2011
    \item CY 2011 reported (incurred) claims
    \item PY 2010 reported (incurred) claims @ 12/31/2011
\end{itemize}

\begin{table}[htbp]
\centering
\begin{tabular}{cccccc}
\toprule
Accident Date & Transaction Date & Claim Status & Payment & Case Reserve \\
\midrule
% Claim 1 (Policy A)
\textbf{Claim 1} (Policy A) & & & & \\
01/01/2010 & 01/20/2010 & Open & \$500 & \$1000 \\
& 02/05/2010 & Open & \$1000 &  \$1000\\
& 01/10/2011 & Closed & \$1500 & \\
\midrule
% Claim 2 (Policy A)
\textbf{Claim 2} (Policy A) & & & & \\
01/15/2011 & 01/25/2011 & Open & \$300 & \$800 \\
& 02/15/2011 & Open & \$200 & \$800 \\
& 08/05/2011 & Open & \$400 & \$500 \\
\midrule
% Claim 3 (Policy C)
\textbf{Claim 3} (Policy C) & & & & \\
04/01/2011 & 09/15/2011 & Open & \$600 & \$1200 \\
& 10/10/2011 & Open & \$700 & \$700 \\
& 11/05/2011 & Open & \$800 & \$400 \\
& 12/10/2011 & Open & \$900 & \\
& 01/05/2012 & Closed & \$1000 & \\
\bottomrule
\end{tabular}
\end{table}

\item 
You know that reported claims for the last few years follow this claims triangle:
\begin{table}[htbp]
    \centering
    \begin{tabular}{|c|c|c|c|}
    \hline
    \multicolumn{4}{|c|}{\textbf{Claims Triangle}} \\ \hline
    \textbf{Accident Year} & \textbf{DY0} & \textbf{DY1} & \textbf{DY2} \\ \hline
    \textbf{2022} & $120$ & $180$ & $240$ \\ \hline
    \textbf{2023} & $150$ & $200$ &  \\ \hline
    \textbf{2024} & $170$ & &  \\ \hline
    \end{tabular}
\end{table}
You also know the following:
\begin{itemize}
\item The permissible loss ratio is $80\%$.
\item The trend factor for exponential claims growth is $\delta = 0.07$.
\item The first two loss development factors for losses are estimated to be $1.35$ and $1.30$.
\item Assume no additional losses from years prior to 2022 and no tail factor.
\end{itemize}
What should the rate be based on the loss cost method for a new one-year policy starting in 2025?

\newpage

\item  
The following earned premiums were calculated for the specific years given in the table:
\begin{table}[h]
\centering
\begin{tabular}{@{}cc@{}}
\toprule
Calendar Year & Earned Premium (\$) \\ \midrule
2017          & 1,800,000           \\
2018          & 2,000,000           \\
2019          & 2,200,000           \\ \bottomrule
\end{tabular}
\end{table}

During this time, rates were subjected to the following rate changes:
\begin{table}[h]
\centering
\begin{tabular}{@{}ccc@{}}
\toprule
Rate Change & Effective Date  & Percentage Change \\ \midrule
1           & September 1, 2017 & 10\%              \\
2           & April 1, 2018    & 15\%              \\ \bottomrule
\end{tabular}
\caption{Rate changes during the calendar years}
\end{table}


Assume that rate changes are applied proportionally to the remaining part of the year from the date they take effect and that all policies are 1-year policies. Other actuaries have determined that expected effective losses for 2020 are 1,600,000 and fixed expenses are 150,000. The permissible loss ratio is 80\%. The current average rate per exposure unit is 1100. Based on this information, use the loss ratio method to determine rates for the 2020 year. 

\item 
You are given the following information:

\begin{table}[htbp]
\centering
\begin{tabular}{@{}ll@{}}
\toprule
\textbf{Item}                          & \textbf{Value}              \\ \midrule
Expected Effective Losses (Trended and Developed)    & \$500,000                   \\
Exposure Units                         & 10,000                      \\
Earned Premium at Current Rates        & \$700,000                   \\
Current Average Premium                & \$70                        \\
Fixed Expenses                         & \$100,000                   \\
Fixed Expenses per Exposure Unit       & \$10                        \\
Permissible Loss Ratio                 & 80\%                        \\ \bottomrule
\end{tabular}
\end{table}

You need to find the new rate based on the loss cost method and the loss ratio method.




   
   \end{enumerate}

\end{document}
