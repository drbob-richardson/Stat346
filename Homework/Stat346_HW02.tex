\documentclass{article}
\usepackage{setspace,fullpage}
\usepackage{xcolor,amsmath}


\begin{document}

\begin{center}
\textbf{Stat 346 Homework \#2}\\
\vspace{0.3in}
\end{center}
\begin{enumerate}
\item
 For a given health insurance policy, the loss amount (expressed in thousands) per year has a pdf of
    \[f(y) = cy(5-y), 0 < y < 4\]
    where $c$ is a constant.
    \begin{enumerate}
        \item Calculate the $99^{th}$ percentile of the loss amount. (Use software to solve the final polynomial) \textcolor{red}{[3.954]}
        \item Calculate the coefficient of variation of the loss amount. \textcolor{red}{[.4257]}
        \item For a policy with a deductible of \$1000, calculate $E(Y^P)$ and $E(Y^L)$. \textcolor{red}{[\$1500, \$1326]}
        \item For a policy with no deductible but with a limit of \$2500, calculate the expected loss amount. \textcolor{red}{[\$1976.84]}
    \end{enumerate}
%\item
%A claim count distribution can be expressed as a modified Poisson distribution. The mean of the Poisson distribution is uniformly distributed over the interval [0,5]. Calculate the probability that there are 2 or more claims. \textcolor{red}{[.6094]}
%\item 
%The number of claims of an insurance coverage follows a Poisson distribution with mean $\lambda$ for each insured. The means $\lambda$ vary by insured and overall follow a gamma distribution. You are given:
%	\begin{enumerate}
%	\item[i]The probability of 0 claims for a randomly selected insured is 0.04.
%	\item[ii] The probability of 1 claim for a randomly selected insured is 0.075.
%	\item[iii] The probability of 2 claims for a randomly selected insured is 0.0879.
%	\end{enumerate}
% Determine the variance of the gamma distribution. \textcolor{red}{[3.119]}
\item
For a frequency distribution in the $(a,b,0)$ class, you are given
	\begin{enumerate} 
	\item[i] $p_{k}=0.0768$
	\item[ii] $p_{k+1}=p_{k+2}=0.08192$
	\item[iii] $p_{k+3}=0.0786432$
	\end{enumerate}
Determine the mean of this distribution by using E(N)=$\frac{a+b}{1-a}$ \textcolor{red}{[8]}
\item 
Claim frequency follows a distribution in the $(a,b,0)$ class. You are given that
	\begin{enumerate}
	\item[i]The probability of 4 claims is 0.066116.
	\item[ii] The probability of 5 claims is 0.068761.
	\item[iii] The probability of 6 claims is 0.068761.
	\end{enumerate}
Calculate the probability of no claims. \textcolor{red}{[.0179]}
\item
A random variable follows a zero-truncated Poisson distribution with $\lambda$=0.8. Calculate the third raw moment of the distribution. \textcolor{red}{[5.869]}

\item A random variable $X$ has a c.d.f. of \[F_X(x) = 1-e^{-x^2/(2\sigma^2)}, ~~ x > 0. \] The 90th percentile is 4.29. What is $\sigma$?

\item For a certain random variable, $X$, the TVaR$_{95}$ is 4 and the VaR$_{95}$ is 3. Determine the difference between the expected value of $X$ and the expected value of the limited loss random variable with a cap at $u=3$. 


\item A random variable $X$ has pdf \[ f_X(x) = \frac{8}{(x+2)^3},~~x>0\] Determine the TVaR$_{99}$. 

\item Suppose $N$ is a counting distribution satisfying the recursive probabilities:
\[\frac{p_{k}}{p_{k-1}} = 0.8 + \frac{3.2}{k}\]
for $k=1,2,\ldots$
Identify the distribution of $N$.

\item Consider the zero-truncated Binomial distribution with probabilities
\begin{eqnarray*}
p_0 &=& 0 \\
p_k &=& C \binom{8}{k} 0.3^k (0.7)^{8-k} , \ \ \text{for} \ \ k=1,...,8
\end{eqnarray*}
Find the value of $C$. Derive the mean and the variance of this distribution.

\end{enumerate} 
\end{document}