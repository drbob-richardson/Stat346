\documentclass{article}
\usepackage{setspace,fullpage,xcolor}
\usepackage{amssymb}
\usepackage{amsfonts}
\usepackage{amsmath}
\usepackage{amsxtra}
\usepackage{hyperref}
\usepackage{tensor}
\usepackage{dsfont}

\begin{document}

\begin{center}
\textbf{Stat 346 Homework \#3}\\
\vspace{0.3in}
\end{center}
\begin{enumerate}
%\item
%For a given policy, there are two types of policy holders. The loss amount of a policy holder from type one is exponentially distributed with a mean of 2. Type 2 losses are exponentially distributed with a mean of 5.
%    \[X_1 \sim \textrm{EXP}(2) \qquad X_2 \sim \textrm{EXP}(5)\]
%Also, assume that $S=X_1 + X_2$, the total loss from a portfolio of one policyholder of type 1 and one of type 2, and that $X_1$ and $X_2$ are independent.
%    \begin{enumerate}
%        \item Calculate $\textrm{TVaR}_{0.95}(X_1)$. \textcolor{red}{[7.99]}
%        \item Calculate $\textrm{TVaR}_{0.95}(X_2)$. \textcolor{red}{[19.98]}
%        \item Calculate $\textrm{TVaR}_{0.95}(S)$.   \textcolor{red}{[22.53]}
%        \item Using the above results, what does that tell you about the subadditivity of $\textrm{TVaR}$ and the practical benefit of diversification?
%        \item Further assume that 30\% of all policy holders are of type 1. \\What is the mean and variance of a loss from a randomly selected policyholder? \textcolor{red}{[4.1, 20.59]}
%    \end{enumerate}
\item
For a commercial fire coverage
	\begin{itemize}
	\item In 2009, loss sizes follow a two-parameter Pareto distribution with parameters $\alpha$=4 and $\theta$.
	\item In 2010, there is uniform inflation at rate $r$.
	\item The 65th percentile of loss size in 2010 equals the mean loss size in 2009.
	\end{itemize}
Determine $r$. \textcolor{red}{[.1107]}
\item
$X$ follows a gamma distribution with parameters $\alpha$=3.5 and $\theta$=100.
$Y=1/X$
\\Evaluate Var($Y$) \textcolor{red}{[$\frac{1}{93750}$]}

\item 
$X$ follows an exponential distribution with mean 50.\\Determine the mean of $X^4$. \textcolor{red}{[150,000,000]}

\item
You are given the following:
	\begin{itemize}
	\item Losses in 1993 follow the density function
	\[f(x) = 3x^{-4}, x\ge 1,\]
	where x=losses in millions of dollars.
	\item Inflation of 10\% impacts all claims uniformly from 1993 to 1994.
	\end{itemize}
Determine the probability that losses in 1994 exceed 2.2 million. \textcolor{red}{[.125]}

\item
Claim sizes in 2010 follow a lognormal distribution with parameters $\mu$=4.5 and $\sigma$=2. Claim sizes grow at 6\% uniform inflation during 2011 and 2012.\\Calculate $f$(1000), the probability density function at 1000, of the claim size distribution in 2012. \textcolor{red}{[.000103]}

\item
   Consider a random variable $Z$ with the following moments:
   \begin{itemize}
     \item $E(Z) = 4.5$
     \item $E(Z^{0.5}) = 2$
     \item $E(Z^2) = 30$
   \end{itemize}
   
   Determine which distribution(s) from the following list best fit these moments:
   \begin{itemize}
     \item Lognormal distribution
     \item Exponential distribution
     \item Weibull distribution
   \end{itemize}


\item 
   For each of the following probability density functions, determine how many (integer) moments exist for the random variable:

   a. $f(x) \propto e^{-2x}, \quad x > 0$

   b. $f(y) = \dfrac{k}{y^3}, \quad y > 1$

   c. $f(z) \propto z^{-4} e^{-3/z}, \quad z > 0$

\item 
   Let $X$ be a random variable following a Weibull distribution with parameters $\tau$ and $\theta$. Prove that if you multiply $X$ by a positive constant $c$, i.e., $Y = cX$, then $Y$ follows a Weibull distribution with parameters $\tau$ and $c\theta$. Show the steps of your proof.
\end{enumerate}


\end{document} 