% This syllabus template was created by:
% Brian R. Hall
% Assistant Professor, Champlain College
% www.brianrhall.net

% Document settings
\documentclass[12pt]{article}
\usepackage[margin=1in]{geometry}
\usepackage[pdftex]{graphicx}
\usepackage{multirow}
\usepackage{setspace, url}
\pagestyle{plain}
\setlength\parindent{0pt}
\usepackage{array}
\usepackage[table]{xcolor}
\geometry{letterpaper, portrait, margin=1in}

\begin{document}

% Course information
\begin{tabular}{ l l }
  \multirow{3}{*} & \LARGE Stat 346 \\\\
  & \LARGE Short Term Actuarial Math \\\\
  & \LARGE MWF 1:00 - 2:00 1159 WVB \\\\
\end{tabular}
\vspace{10mm}

% Professor information
\begin{tabular}{ l l }
  \multirow{6}{*} & \large Professor: Robert Richardson \\
  & \hspace{5 mm} \large email: richardson@stat.byu.edu \\
  & \hspace{5 mm} \large Office Location: 2186 WVB \\
  & \hspace{5 mm} \large Zoom Room: 8014223736 \\
  & \hspace{5 mm} \large Office Hours: Wednesday 9:15 to 11:15 \\
  & \hspace{5 mm} \large 801-422-3736 \\
\end{tabular}
\vspace{5mm}

% TA information
\begin{tabular}{ l l }
  \multirow{6}{*} & \large TA: Davis Dowdle \\
  & \hspace{5 mm} \large email: davis.dowdle@gmail.com \\
  & \hspace{5 mm} \large Office Location: 1151 WVB  \\
  & \hspace{5 mm} \large Office Hours: M: 3-5; T/Th: 12-2; W: 2-4 \\
\end{tabular}
\vspace{5mm}


% Course details
\textbf { \\ Course Description:} Prepare for the short term portion of Actuarial SOA Exam FAM and elements of CAS exam 5 and CAS exam MAS-I. The topics of study include modeling frequency and severity of claims, parameter estimation in those models, and assessing the appropriateness of these models.  

\vspace{.5 cm}

\textbf { Texts:} \emph{Loss Models: From Data to Decisions},  5\textsuperscript{th} Edition

\hspace{1 cm} \small{\emph{Introduction to Ratemaking and Loss Reserving for Property and Casualty Insurance}, 4\textsuperscript{th} Edition }

\vspace{.5 cm}


\textbf{Course Management:} Learning Suite \\

\vspace{.5 cm}

% I recommend using \newpage here if necessary
\textbf { Grade Distribution:} \\
\hspace*{40mm}
\begin{tabular}{ l l }
Homework & 20\% \\
Midterm & 30\% \\
Final Exam  & 50\% 
\end{tabular} \\\\

\vspace{.5 cm}


\textbf{SOA University Earned Credit}
This course qualifies for the SOA University Earned Credit program for exam FM. If you score at least 85\% in this course, you will receive credit for the short term portion of exam FAM. To get full credit for exam FAM you must also take and pass Stat 344. 

\newpage

\textbf { Homework:}
Homework is due on the listed due date. There will be about one homework per week and due dates may vary. Solutions to many of the homework questions will be available online after the homework is due. 

\vspace{.5 cm}


\textbf { Exams:}
There is one midterm and one final exam. The midterm will be. in class covering two class periods: February 26th and February 28th. The final exam will be in class at our scheduled final exam time, Monday April 22nd at 11 AM. Exams are closed books and closed notes. The final exam will be cumulative. 

\vspace{.5 cm}


\textbf{ Discord:} We will have a Discord server for class. Feel free to ask questions on homework or other topics you are needing help with or any clarifications. The following link will be active the first 7 days of the semester: \url{https://discord.gg/SwG6P4Np}. After that, or if it doesn't work, email Dr. Richardson for a new link. 

\vspace{.5 cm}


\textbf{ Prayer Schedule:} If you are comfortable with praying in class, go ahead and sign up for days to pray in class at this link: \url{https://docs.google.com/spreadsheets/d/1hapGrSvTHy44J0F99hUsTI9mjD_ICtTyDtmFLZAUI_g/edit?usp=sharing}. This link will also be available on Learning Suite. 

%\textbf {\large Teaching Assistants:} There are three teaching assistants for this class:
%\begin{table}[htp]
%\begin{tabular}{ l l }
%Jared Hoff & hoff.jared@gmail.com \\
%James Marriott & james.mahlan.marriott@gmail.com \\
%Allen Zhou & allen.z@byu.edu \\
%\end{tabular} 
%\end{table} \\
%
%They will have office hours in the actuarial lab. Their office hour times are
%\begin{table}[htp]
%\begin{tabular}{|c|c|c|c|c|c|}
%\hline
%Time & M & T & W & Th & F \\ \hline
%8:00 & & & & JH & \\ \hline
%9:00 & & & & JH & \\ \hline
%10:00 & JH & & & JH & \\ \hline
%11:00 & JH & & JH & & JH \\ \hline
%12:00 & JM & & JM & & JH \\ \hline
%1:00 & & & JM & &  \\ \hline
%2:00 & & AZ & & AZ &  \\ \hline
%\end{tabular} \\\\
%\end{table}


\vspace{.5 cm}

\textbf{Expected Schedule:}

\begin{tabular}{ | m{3cm} | m{12cm} | }
\hline
\rowcolor{gray!25}
\textbf{Week} & \textbf{Topic} \\
\hline
Week 1 & Random Variables (Chapters 2 \& 3)  \\
\hline
Week 2 & Claims Frequency Distribution Models (Chapter 6)  \\
\hline
Week 3 &  Claim Severity Models (Sections 5.3-5.4)  \\
\hline
Week 4 & Creating New Distributions (Section 5.2)  \\
\hline
Week 5 & Coverage Modifications (Chapter 8)  \\
\hline
Week 6-7 & Aggregate Loss Models (Chapter 9)  \\
\hline
Week 8-9 & Parameter Estimation (Chapters 13-15)  \\
\hline
Week 10-11 & Credibility (Chapters 17-19)  \\
\hline
Week 12 & Reserving \\
\hline
Week 12 &  Ratemaking \\
\hline
Week 13 & Options \\
\hline
\end{tabular}

Book chapters are for the textbook labeled \emph{Loss Models}. 

\vspace{1 cm}

% College Policies
\textbf {\large Honor Code:} 
% This should be specific to your instituition, an example is provided.
In keeping with the principles of the BYU Honor Code, students are expected to be honest in all of their academic work. Academic honesty means, most fundamentally, that any work you present as your own must in fact be your own work and not that of another. Violations of this principle may result in a failing grade in the course and additional disciplinary action by the university. Students are also expected to adhere to the Dress and Grooming Standards. Adherence demonstrates respect for yourself and others and ensures an effective learning and working environment. It is the university's expectation, and my own expectation in class, that each student will abide by all Honor Code standards. Please call the Honor Code Office at 422-2847 if you have questions about those standards.

\vspace{.5 cm}


\textbf {\large Sexual Harassment:} 
Title IX of the Education Amendments of 1972 prohibits sex discrimination against any participant in an educational program or activity that receives federal funds. The act is intended to eliminate sex discrimination in education and pertains to admissions, academic and athletic programs, and university-sponsored activities. Title IX also prohibits sexual harassment of students by university employees, other students, and visitors to campus. If you encounter sexual harassment or gender-based discrimination, please talk to your professor or contact one of the following: the Title IX Coordinator at 801-422-2130; the Honor Code Office at 801-422-2847; the Equal Employment Office at 801-422-5895; or Ethics Point at http://www.ethicspoint.com, or 1-888-238-1062 (24-hours).

\vspace{.5 cm}


\textbf {\large Student Disability:} 
Brigham Young University is committed to providing a working and learning atmosphere that reasonably accommodates qualified persons with disabilities. If you have any disability which may impair your ability to complete this course successfully, please contact the University Accessibility Center (UAC), 2170 WSC or 422-2767. Reasonable academic accommodations are reviewed for all students who have qualified, documented disabilities. The UAC can also assess students for learning, attention, and emotional concerns. Services are coordinated with the student and instructor by the UAC. If you need assistance or if you feel you have been unlawfully discriminated against on the basis of disability, you may seek resolution through established grievance policy and procedures by contacting the Equal Employment Office at 422-5895, D-285 ASB.

\vspace{.5 cm}


\textbf {\large Mental Health Services:} 
Barriers to learning are created by stress, anxiety, family and relationship concerns, and personal crises. If stressful life events or mental health concerns are inhibiting your ability to participate in daily activities or leading to diminished academic performance, please contact the BYU Counseling and Psychological Services (CAPS; 1500 WSC, 801-422-3035, caps.byu.edu). CAPS provides individual, couples and group counseling to students. These services are confidential and are provided by the university at no added cost to you. Professional psychologists and counselors who specialize in helping college students are available 24-hours a day to assist students in crisis; if you have an emergency during non- business hours (5pm-8am), please contact BYU Police Dispatch (801-422-2222) who will put you in touch with a counselor.

\end{document}



